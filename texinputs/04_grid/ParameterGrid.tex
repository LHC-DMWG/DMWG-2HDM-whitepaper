
%\paragraph{Logic of how we proceeded}
%
%\begin{itemize}
%\item Starting from benchmark 3 of \cite{Bauer:2017ota}
%\item Mapping the kinematics and sensitivity of the model by scanning some of the
%various parameters
%\item Checking whether other existing models can be rescaled
%\end{itemize}
%
%\subsubsection{Results of studies}
%
%Each of the signatures should have the following plots in the planes
%of the final recommendation: 
%\begin{itemize} 
%\item efficiency at parton level with simplified, published cuts
%\item total and fiducial cross-section at parton level 
%\item 2 - 3 kinematic plots of what has been scanned that are most representative for the analysis (here the analysers decide, then we harmonize at the end)
%\end{itemize} 
%
%Signatures:
%
%\begin{itemize}
%
%\item{Mono-Z (lep/had)}
%
%\item{MonoH$\rightarrow$bb}
%
%\item{Monojet}
%
%\item{ttbar+MET}, with specific discussion about rescaling
%
%\item{other signatures who have not yet presented at public meetings, in ATLAS and CMS}
%
%\end{itemize}

%\subsection{Parameter scan}

The studies in the previous section show that varying most of the model parameters lead to non-trivial modifications of the for the H+\MET and Z+\MET searches. 
We decide to investigate the model parameter space through two-dimensional and one-dimensional scans of five parameters: the light pseudoscalar mass (\ma), the heavy pseudoscalar mass (\mA) that we set equal to the mass of the heavy and charged Higgs bosons (\mA = \mH = \mHc), the mixing angle $\sinp$, the ratio of VEVs of the Higgs doublets $\tanb$ and the dark matter particle mass \mDM. 
The benchmark model points that have been agreed within the DMWG and are suggested here do not provide an exhaustive scan the entire parameter space of this model, but highlights many of the features that are unique of this model and showcases the complementarity of the various signatures. 

\paragraph{Scan in the \ma, \mA = \mH = \mHc plane}

The main parameter grid proposed to investigate this model with LHC data spans combinations of the light pseudoscalar mass (\ma) and the heavy pseudoscalar mass (\mA) plane, fixing \mA = \mH = \mHc. The mixing angle $\sinp$ is fixed to 0.35, to evade precision constraints. $\tanb$ is fixed to unity to obtain a mixture of resonant and non-resonant processes for the H+\MET and Z+\MET searches. The DM particle mass is fixed to 10 GeV, to obtain cross-sections that are sufficiently large to be probed by Run-2 LHC searches. The spacing of the grid in \ma and \mA is left to the individual searches. The parameters $\sinp$, $\tanb$ and \mDM are scanned separately.

\paragraph{Scan in the \ma, $\tanb$ plane}

A two-dimensional scan in the \ma, $\tanb$ plane, fixing \mA = \mH = \mHc = 600 GeV, is used to emphasize the complementarity of the H+\MET and Z+\MET searches with the heavy flavor + \MET searches. The scan in \ma includes masses between 10 and 350 GeV, while the $\tanb$ scan includes $\tanb$ = 50, 45, 40, 35, 30, 25, 20, 15, 10, 5, where the high-$\tanb$ points are of primary interest for the heavy flavor searches. 
%It has been shown in \autoref{sec:DMHF} that the kinematics corresponds to a mixture of the previous DMF models.

\paragraph{Scans in $sin_{\theta}$}

Two one-dimensional scans in $sin_{\theta}$ are also suggested for further comparison of the H/Z+\MET and $b\bar{b}$+\MET analyses. In the first scan, resonant processes dominate with \mA = \mH = \mHc = 600 GeV and \ma=200 GeV, while in the second scan \mA = \mH = \mHc = 1000 GeV and \ma=350 GeV. For both scans, $\tanb$ and the DM mass are fixed to $\tanb$=1 and \mDM = 10 GeV. 
%TODO: add more info on what is expected here

\paragraph{Scan in \mDM}

A one-dimensional scan in \mDM spanning from 1 GeV to 500 GeV, with fixed \mA=\mH=600 and \ma=250 GeV, is also suggested to connect this model to a standard cosmological history. Even though the model points with where the DM particle has a mass above 100 GeV are not within immediate reach of Run-2 searches, the measured relic density is satisfied by this model at values of DM mass around 100 GeV, as shown in \autoref{sec:relic}.

%
% (it is expected that the bbar+MET
%analysis will only have to rescale previous models/cross-sections)
%{[}2{]}: - mH± = mA = mH = 600GeV , ma = 200GeV, tanBeta=1 - mH± = mA =
%mH = 1000GeV , ma = 350GeV, tanBeta=1
%
%
%
%\item a two-dimensional scan in the ma − tanBeta plane, for
%comparison with the ttbar+MET / bbar+MET analyses. In this case, the
%charged Higgs mass (mH+/-), the heavy pseudoscalar mass (mA) and the
%heavy Higgs mass (mH) should be fixed to 600 GeV. This scan includes points: 
%50, 45, 40, 35, 30, 25, 20, 15, 10, 5
%for M(a) masses between 10 and 350 GeV. The high-tanBeta points would be
%of primary interest to the HF + DM searches. Uli's studies have shown
%that one can simply reweight the existing tt+DM/bb+DM models from DMF to
%the new 2HDM+PS cross sections; full simulation of the newly proposed
%2HDM+PS points is not required.
%
%
%
%%what changes
%
%main changes in the kinematic distribution for this model occur when varying the 
%
%
%Based on the studies in the previous section, the main changes in the kinematics occur in the mixing angle 
%
%plane to be probed as benchmark is that of the 
%?
%
%\begin{itemize}
%
%\item 
%
%\item 
%\end{itemize}
%
%
%In order to explore changes in complementarity with different
%analyses and kinematics, this should be complemented by:
%
%\begin{itemize}
%
%
%
%
%\end{itemize}

%The PDF recommended is five-flavor. ATLAS will use the NNPDF3.0
%PDF set. Some text by Fabio Maltoni and Ulrich Haisch can be found in the texinputs\_app folder.  

