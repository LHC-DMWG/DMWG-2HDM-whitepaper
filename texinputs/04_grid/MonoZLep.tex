\paragraph{Technical setup}
Simulated event samples for the mono-Z signature are produced with Madgraph5\_aMC@NLO version 2.4.3, interfaced with Pythia version 8.2.2.6 for parton showering. The NNPDF3.0 PDF set is used at LO precision with the value of the strong coupling constant set to $\alpha_{S}(M_{Z}) = 0.130$ (NNPDF30\_lo\_as\_0130). Only contributions from gluon-gluon initial states and \lp\lm$\chi\overline{\chi}$ final states are considered, where l = e or $\mu$. No additional matrix element partons are considered and diagrams with an intermediate s-channel SM Higgs boson are explicitly rejected to increas the calculation efficiency (generate g g $>$ xd xd~ l+ l- / h1). 


\paragraph{Event selection}
Three consecutive stages of event selection are considered:
\begin{itemize}
\item Inclusive: Lepton \pt and $\eta$ requirements corresponding to the typical experimental trigger acceptance are applied.

\item Preselection: A dilepton candidate with an invariant mass in a window around the Z mass is required, and a minimum transverse momentum of the $\chi\overline{\chi}$ system is required.

\item Final selection: Requirements on the main variables used in the relevant analyses are added: The angular separation in the transverse plane between the $\chi\overline{\chi}$ and \lp\lm systems $\Delta\Phi(ll,\MET)$, the relative transverse momentum difference between them $|p_{T,ll} - \MET|/p_{T,ll}$ and the angular separation between the leptons $\Delta R(ll)$. Additionally, the \MET requirement is tightened.
\end{itemize}

The exact event selection criteria are listed in Tab.~\ref{tab:monozll_selection}.

\begin{table}
\centering
\caption{Event selection requirements for the analysis of the Mono-Z signature with leptonic Z decays.
        The requirements are inspired to follow those used in typical experimental analyses.}
\begin{tabular}{c | c |r l}
Selection stage & Quantity & Requirement \\\hline


\multirow{ 2}{*}{Inclusive}         & lepton $\left|\eta\right|$                    & $< 2.5$ \\
                                    & leading (trailing) lepton \pt                 & $> 25 (20)$ GeV \\\hline

\multirow{ 2}{*}{Preselection}      & $\left|m_{ll}-m_{Z,\mathrm{nominal}}\right|$  & $< 15$ GeV\\
                                    & \MET                                          & $> 40$ GeV \\\hline

\multirow{ 3}{*}{Final selection}   & $\Delta\Phi(ll,\MET)$                         & $>2.7$\\
                                    &$|p_{T,ll} - \MET|/p_{T,ll}$                   & $<0.4$\\
                                    &  $\Delta R(ll)$                               & $<1.8$\\
\end{tabular}


\label{tab:monozll_selection}

\end{table}


\paragraph{Results}
The overall cross-sections in the \tanb and mass scans are shown in Fig.~\ref{fig:xs_inclusive}.
In the mass scan, maximal cross-sections are observed for the region of $\ma < \mA$ for values of $\ma\gtrsim100$ GeV. Towards higher values of both \ma and \mA, the cross-sections fall off, reaching values smaller than $1$ fb at $\ma\approx450$ GeV or $\mA\approx1.1$ TeV. In the $\ma\approx\mA$-region, the cross-section is suppressed by destructive interference. For the region with inverted mass hierarchy $\ma>\mA$, cross-sections of the order of multiple fb are observed, as long as $|\ma-\mA|$ remains sufficiently large.
In the \tanb scan, cross-sections smoothly fall with increasing $\ma$ as well as $\tanb$. Cross-sections are typically larger than 1 fb up to $\tanb\approx5$. The dependence on $\ma$ is modulated by the value of \tanb: Crossing the $\ma$ range from $100$ to $400$ GeV, cross-sections are reduced by a factor $\approx7$ for small $\tanb\approx1$, but only a factor $\approx2$ for higher values of $\tanb\approx5$.


To assess the kinematic behavior of the signal, the distributions of the kinematic variables that are most relevant to the Mono-Z signature are studied as a function of the model parameters.

The distribution of the invariant masses of the dilepton and $\chi\chi$ systems are shown in Fig.~\ref{fig:monoz_kin_inclusive}. Independent of the of parameters, the dilepton mass spectrum is centered at the Z peak, without any nonresonant contribution. The $M_{\chi\chi}$ distribution illustrates the signal contributions from different diagrams. For $\mA>\ma$, DM is dominantly produced from on-shell $a$ boson production. In the inverted mass region $\mA<\ma$, the situation is reversed, and $A$ diagrams dominate.

After applying the preselection requirements, the distributions of kinematic variables are shown in Fig.~\ref{fig:monoz_kin_presel}. In the region of $\mA>\ma$, distinct Jacobian peaks are visible in the distributions of the mediator \pt, the width of which generally increases with the $\ma$ and $\mA$. Significant portions of the spectrum are situated at relatively high boosts ($\MET>200 GeV$), which is more easily accessible experimentally.
This behavior is constrasted by the distributions in the inverted mass regions, which show nearly no distinct features and are mainly located at low mediator \pt. For $\mA~\ma+m_{Z}$, both the a and Z bosons are produced close to at rest, leading to an event population with overall low boost. These qualitative trends are consistend between the observables studied here.
Finally, the distributions of the $\MET$ and $\MT$ variables after final selection are shown in Fig.~\ref{monoz_kin_final}. Traditionally, the Mono-Z search has relied on the $\MET$ distribution for signal extraction. While the presence of the Jacobian peak structure in the distribution facilitates signal-background separation, it may be beneficial to also consider the $\MT$ distribution. Although only transverse information is available, the resonant structure of the signal is significantly enhanced in the $\MT$ variable, which may enhance the sensitivity of a specialized search strategy.

The \tanb and $\sin(\theta)$ variables have no effect on the distributions of kinematic variables (Fig.~\ref{fig:monoz_kin_tanb_sintheta}).


\begin{figure}
\centering
\includegraphics[width=\textwidth]{texinputs/04_grid/figures/monoz/leptonic/xs_2d_inclusive_26300.pdf}
\caption{Inclusive cross-sections for $pp\rightarrow \lp\lm\chi\overline{\chi}$ in the \ma-\mA scan. Maximal cross-sections are observed for the region of $\ma < \mA$ for values of $\ma\gtrsim100$ GeV.  In the $\ma\approx\mA$-region, the cross-section is suppressed by destructive interference. Finally, for a region with inverted mass hierarchy $\ma>\mA$, cross-sections of the order of multiple \ifb are observed, as long as $|\ma-\mA|$ remains sufficiently large.}
\end{figure}


\begin{figure}
\centering
\includegraphics[width=\textwidth]{texinputs/04_grid/figures/monoz/leptonic/tanbma_xsec_ll.pdf}
\caption{Inclusive cross sections for the \ma-tanBeta scan.  \mA fixed to 600 GeV and sinTheta to 0.35} 
\end{figure}

\begin{figure}
\includegraphics[width=\textwidth]{texinputs/04_grid/figures/monoz/leptonic/acceptance.pdf}
\includegraphics[width=\textwidth]{texinputs/04_grid/figures/monoz/leptonic/xs_2d_dmwg-final_26300_yield40fb.pdf}
\caption{Acceptance and event yields in the  \ma-\mA plane after applying the final selection. Event yields assume an integrated luminosity of $40~\ifb$. The acceptance is maximal for $\mA > \ma$, where it reaches 50 \%. In the inverted mass region $\mA < \ma$, lower values of 10-30\% are observed. In the intermediate region around $\mA \approx \ma + \mZ$, the acceptance is strongly suppressed as the a and Z bosons are produced approximately at rest.}
\label{fig:monoz_ll_acceptance}
\end{figure}

\begin{figure}
\includegraphics[width=\textwidth]{texinputs/04_grid/figures/monoz/leptonic/tanbma_ae_ll.pdf}
\caption{Acceptances across the \ma-tanBeta scan.  Acceptance is flat over tanBeta for constant values of \ma.}
\end{figure}

\begin{figure}
\includegraphics[width=\textwidth]{texinputs/04_grid/figures/monoz/leptonic/tanbma_yield_ll.pdf}
\caption{Event yield in the \ma-tanBeta grid, for an integrated luminosity of $40~\ifb$.  The number of expected events diminshes with increasing tanBeta and \ma.  \mA fixed to 600 GeV and sinTheta to 0.35}
\end{figure}


\begin{figure}
\includegraphics[width=\textwidth]{texinputs/04_grid/figures/monoz/leptonic/mA_Scan.pdf}
\caption{The position of the Jacobian peak in the \MET distribution depends on the difference between \mH and \ma.  For fixed values of \ma and \mA = \mH, increaseing \mA shifts the peak towards higher enrgies, and decreasing \mA shifts it lower.  For small mass splittings between \mH and \ma, most events will fail to pass the \MET selection criteria.}   
\end{figure}

\begin{figure}
\includegraphics[width=0.45\textwidth]{texinputs/04_grid/figures/monoz/leptonic/SinThetaScan_mH600ma250_MET.pdf}
\includegraphics[width=0.45\textwidth]{texinputs/04_grid/figures/monoz/leptonic/SinThetaScan_mH700ma350_MET.pdf}
\caption{Performing one dimensional scans of sinTheta shows that it has little impact on the events' kinematic distributions.  The first scan is performed at \mA = 600 GeV and \ma = 250 GeV, the second at \mA = 700 GeV \ma = 350 GeV.  In both cases tanBeta = 1.0.}
\end{figure}


\begin{figure}
\centering
\includegraphics[width=0.45\textwidth]{texinputs/04_grid/figures/monoz/leptonic/inclusive_h_mz_lep.pdf}
\includegraphics[width=0.45\textwidth]{texinputs/04_grid/figures/monoz/leptonic/inclusive_h_m_med_dm.pdf}
\caption{Distributions of the invariant mass of the dilepton (left) and $\chi\overline{\chi}$ systems (right) with no selection applied in addition to the generation cuts. The $M_{ll}$ distribution is centered around the Z boson mass independent of the chosen parameter point, indicating that there is no contribtion from $\gamma*$ exchange. The $M_{\chi\overline{\chi}}$ distribution }
\end{figure}


\begin{figure}
\centering
\includegraphics[width=0.45\textwidth]{texinputs/04_grid/figures/monoz/leptonic/presel_h_balance.pdf}
\includegraphics[width=0.45\textwidth]{texinputs/04_grid/figures/monoz/leptonic/presel_h_dphi_met_ll.pdf}
\includegraphics[width=0.45\textwidth]{texinputs/04_grid/figures/monoz/leptonic/presel_h_dr_ll.pdf}
\caption{Distributions of the main selection variables after preselection: \pt balance (top panel), $\Delta\Phi$ (middle) and $\Delta R$ (bottom). The shown parameter points illustrate the different qualitative behavior in the three different mass regions. }
\end{figure}


\begin{figure}
\centering
\includegraphics[width=0.45\textwidth]{texinputs/04_grid/figures/monoz/leptonic/dmwg-final_h_pt_med_dm.pdf}
\includegraphics[width=0.45\textwidth]{texinputs/04_grid/figures/monoz/leptonic/dmwg-final_h_mt_total.pdf}
\caption{\MET and  MT distributions in the signal region. The \MET distribution shows a Jacobian structure in the $\mA > \ma$ regime, the location of which strongly depends on $\mA$. In the region of inverted mass hierarchy $\mA < \ma$, the spectrum is less structured and does not fall off as steeply towards higher values. For a small mass splitting of $\ma-\mA\approx M_{Z}$, the spectrum is shifted to much lower values of \MET. The MT distribution allows to access the resonant nature of the process. Clear mass peaks are present for the normal mass hierarchy. In the inverted region, the MT distribution is more sensitive to the mass difference $\ma-\mA$ than the \MET distribution, allowing to differentiate between signal hypotheses that give near-identical \MET distributions. }
\end{figure}


Equation for significance \cite{Cowan:2012}

\begin{equation}
\label{eqn:significance_wsyst}
Z_A = \sqrt{ 2 \cdot \bigg( (s+b) \ln[\frac{ (s+b) (b+\sigma_b^2) } {b^2 + (s+b) \sigma_b^2} ]- \frac{b^2}{\sigma_b^2} \ln[1 + \frac{\sigma_b^2 s}{b(b+\sigma_b^2)} ] \bigg) }
\end{equation}


\begin{figure}
\centering
\includegraphics[width=0.45\textwidth]{texinputs/04_grid/figures/monoz/leptonic/mAma_Significance_ll.pdf}
\includegraphics[width=0.45\textwidth]{texinputs/04_grid/figures/monoz/leptonic/tanbma_Significance_ll.pdf}
\caption{Expected signficicances are calculated using published background estimates and assuming a reconstruction efficiency of 75\%.  The LHC is expected to be sensitive to regions with significances greater than 2.}
\end{figure}

