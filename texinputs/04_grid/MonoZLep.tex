Mono-Z analyses use events with a Z boson recoiling symmetrically against \MET to detect the production of invisible particles.
In previous LHC analyses~\cite{Aaboud:2017bja,Sirunyan:2017qfc}, the DM interpretations of the analysis results have focused on either invisible decays of the SM-like Higgs bosons or topologies where the Z boson is produced as initial-state radiation (ISR) off a quark. The ISR-based topologies generically favor radiation of a gluon or photon rather than a massive gauge boson, thus limiting the discovery sensitivity of a Z-based approach compared to monojet and mono-photon searches. In contrast, the model studied in this document generates the mono-Z signature dominantly via the all-bosonic H-a-Z vertex, which can lead to enhancements in the mono-Z sensitivity compared to jet and photon signatures. In this section, the behavior and experimental accessibility of the model in Z+\MET events is studied.

\paragraph{Technical setup}
Simulated event samples for the leptonic mono-Z signature are produced with Madgraph5\_aMC@NLO version 2.4.3, interfaced with Pythia version 8.2.2.6 for parton showering. The NNPDF3.0 PDF set is used at LO precision with the value of the strong coupling constant set to $\alpha_{S}(M_{Z}) = 0.130$ (NNPDF30\_lo\_as\_0130). A five flavor scheme with a massless b-quark is used.  Only contributions from gluon-gluon initial states and \lp\lm$\chi\overline{\chi}$ final states are considered, where l = e or $\mu$.  The $bb$ initiated ME contribution is negligible for the range of \tanb values studied.  To increase calculation efficiency diagrams with an intermediate s-channel SM Higgs boson are explicitly rejected (generate g g $>$ xd xd~ l+ l- / h1).


\paragraph{Event selection}
Three consecutive stages of event selection are considered:
\begin{itemize}
\item Inclusive: Lepton \pt and $\eta$ requirements corresponding to the typical experimental trigger acceptance are applied.

\item Preselection: A dilepton candidate with an invariant mass in a window around the Z mass is required, and a minimum transverse momentum of the $\chi\overline{\chi}$ system is required.

\item Final selection: Requirements on the main discriminating variables used in the relevant analyses are added: The angular separation in the transverse plane between the $\chi\overline{\chi}$ and \lp\lm systems $\Delta\Phi(ll,\MET)$, the relative transverse momentum difference between them $|p_{T,ll} - \MET|/p_{T,ll}$ and the angular separation between the leptons $\Delta R(ll)$. Additionally, the \MET requirement is tightened.
\end{itemize}

The exact event selection criteria are listed in Tab.~\ref{tab:monozll_selection}.

\begin{table}
\centering
\caption{Event selection requirements for the analysis of the Mono-Z signature with leptonic Z decays.
        The requirements are inspired to follow those used in typical experimental analyses.}
\begin{tabular}{c | c |r l}
Selection stage & Quantity & Requirement \\\hline


\multirow{ 2}{*}{Inclusive}         & lepton $\left|\eta\right|$                    & $< 2.5$ \\
                                    & leading (trailing) lepton \pt                 & $> 25 (20)$ GeV \\\hline

\multirow{ 2}{*}{Preselection}      & $\left|m_{ll}-m_{Z,\mathrm{nominal}}\right|$  & $< 15$ GeV\\
                                    & \MET                                          & $> 40$ GeV \\\hline

\multirow{ 3}{*}{Final selection}   & $\Delta\Phi(ll,\MET)$                         & $>2.7$\\
                                    &$|p_{T,ll} - \MET|/p_{T,ll}$                   & $<0.4$\\
                                    &  $\Delta R(ll)$                               & $<1.8$\\
                                    &  \MET                                         & $>80$ GeV \\ 
\end{tabular}


\label{tab:monozll_selection}

\end{table}


\paragraph{Cross-sections, kinematic distributions, and acceptances}
The overall cross-sections in the \tanb and mass scans are shown in Fig.~\ref{fig:monoz_ll_xs_inclusive}.
In the mass scan, maximal cross-sections are observed for the region of $\ma < \mA$ for values of $\ma\gtrsim100$ GeV. Towards higher values of both \ma and \mA, the cross-sections fall off, reaching values smaller than $1$ fb at $\ma\approx450$ GeV or $\mA\approx1.1$ TeV. In the $\ma\approx\mA$-region, the cross-section is suppressed by destructive interference. For the region with inverted mass hierarchy $\ma>\mA$, cross-sections of the order of multiple fb are observed, as long as $|\ma-\mA|$ remains sufficiently large.
In the \tanb scan, cross-sections smoothly fall with increasing $\ma$ as well as $\tanb$. Cross-sections are typically larger than 1 fb up to $\tanb\approx5$. The $\ma$ dependence is modulated by the value of \tanb: Crossing the $\ma$ range from $100$ to $400$ GeV, cross-sections are reduced by a factor $\approx7$ for small $\tanb\approx1$, but only a factor $\approx2$ for higher values of $\tanb\approx5$.
In the \sinp scan shown in Figure \ref{fig:monoz_ll_sinp_scan_xsec}, cross sections depends on whether or not the $a \rightarrow \ttbar$ decays are accesible.  
For $\ma < 350$ GeV they are not accesible and cross section strictly increases with \sinp.  For $\ma > 350 \GeV$, the $a \rightarrow \ttbar$ decays become possible causing the cross section to decrease for large values of \sinp.

%For \ma <350 GeV they are not accesible,  and only the heavy scalar's branching fraction to $aZ$ is relevant.  This branching fraction stricly increases with \sinp.  For \ma > 350 \GeV, the branching fraction of $a \rightarrow \chi \chi$ also becomes important, and there is a turnover point were the cross section decreases.

To assess the kinematic behavior of the signal, distributions of kinematic variables most relevant to the Mono-Z signature are studied as a function of the model parameters.  Inclusive distributions of the invariant masses of the dilepton and $\chi\chi$ systems are shown in Fig.~\ref{fig:monoz_kin_inclusive}. Independent of the parameters, the dilepton mass spectrum is centered at the Z peak, without any nonresonant contribution. The $M_{\chi\chi}$ distribution illustrates the signal contributions from different diagrams. For $\mA>\ma$, DM is dominantly produced from on-shell $a$ boson production. In the inverted mass region $\mA<\ma$, the situation is reversed, and $H$ diagrams dominate.

After applying the preselection requirements, scans of the model parameters are performed.  For this paper a subset of models are studied where \mA is degenerate with \mH.  Similar to mono-h, for mono-Z in the region $\mA>\ma$ a resonantly produced heavy scalar, $H$, decaying to $aZ$ produces a characteristic Jacobian peak in the \MET distribution.  The location of the peak depends on \mH and \ma as given by equation \ref{eq:monoZ_jacobian_peak}, and the peak's width generally increases with values of  $\ma$ and $\mA$.  

\begin{align}
E^{\mathrm{miss},max}_T \approx \frac{\sqrt{\left(\mH^2 -\ma^2 -M_{Z}^2\right)^2 - 4 \ma^2 M_{Z}^2}}{2\mH}\,.
\label{eq:monoZ_jacobian_peak}
\end{align}

Figure \ref{fig:monoz_ll_mA_scan} shows the Jacobian peaks for increasing values of \mH and fixed \ma.  Significant portions of the spectrum are situated at relatively high boosts ($\MET>\unit[200]{GeV}$), which is more easily accessible experimentally.  This behavior is contrasted by the distributions in the inverted mass regions, which have nearly no distinct features and are mainly located at low mediator \pt. For $\mA\approx\ma+m_{Z}$, both the a and Z bosons are produced approximately at rest, leading to an event population with overall low boost. These qualitative trends are consistent with the distributions of the other main selection variables as shown in Fig.~\ref{fig:monoz_kin_presel}.

Scans of the \tanb and \sinp parameters show they have minimal effect on the kinematic distributions (Fig.~\ref{fig:monoz_kin_tanb_sintheta}).  For small values of $\tanb$ there is a slight softening and broadening of the \MET distribution caused by the increased contribution from non-resonant $z+a$ production.

For \mDM (Fig.~\ref{fig:dm_scan_ll}, in the region where $\mDM < \frac{\ma}{2}$, DM mass has no effect on cross section or the kinematic distributions.  In the off-shell region where  $\mDM > \frac{\ma}{2}$ cross section steeply drops and the \MET distribution becomes less structured.


%After applying the preselection requirements, the distributions of kinematic variables are shown in Fig.~\ref{fig:monoz_kin_presel}. 
%In the region of $\mA>\ma$, distinct Jacobian peaks are visible in the distributions of the mediator \pt, the width of which generally increases with values of $\ma$ and $\mA$. 


Finally, the distributions of the $\MET$ and $\MT$ variables after final selection are shown in Fig.~\ref{fig:monoz_kin_final}. Traditionally, the Mono-Z search has relied on the $\MET$ distribution for signal extraction. While the presence of the Jacobian peak structure in the distribution facilitates signal-background separation, it may be beneficial to also consider the $\MT$ distribution. Although only transverse information is available, the resonant structure of the signal is significantly enhanced in the $\MT$ variable, which may enhance the sensitivity of a specialized search strategy.

Acceptances for the mass and \tanb scans are shown in Figures \ref{fig:monoz_ll_acceptance} and \ref{fig:monoz_ll_tanbma_acceptance}.  
In the mass scan, for points where the Jacobian peak is below the analysis's \MET cut, from $\ma = 100$ $\mA = 200$ to $\ma = 300$ $\mA = 400$, 
acceptance drops sharply approaching zero.  Above this region, acceptance levels are grouped into bands for fixed values of $\mA-\ma$. 
With increasing $\mA-\ma$ acceptances increases, gradually plateauing to a maximum value of 50\%.  In the inverted mass region, 
acceptances are generally lower than the rest of the mass scan, but for light values of \ma can reach values as large as 30\%.
In the \tanb scan, acceptances are largely independent of \tanb and constant for equal values of \ma.  For small values of \tanb and
$\ma < 350$ \GeV  there is slight drop in the acceptance caused by the softer \MET distribution from non-resonant decays.  


\paragraph{Expected significance}

The expected sensitivity of the Mono-Z(ll) channel to 2HDM+a models is approximated using generator level signal samples and background estimates from recent $Z(\ell \ell) + \MET$ searches using 36.1 \ifb of 13 TeV data \cite{Aaboud:2017bja}.  For signal events a reconstruction efficiency of 75\% is assumed, and to be consistent with the background estimates, the same selection cuts as \cite{Aaboud:2017bja} are used.  Signal and background are binned in \MET and a conservative background systematic of 20\% is assumed for $\MET < 120$ \GeV and 10\% above.

Total significance is defined as the per bin significances summed in quadrature.

\begin{equation}
\mathcal{S} = \sqrt{\sum_{bin} (Z^\prime_{bin})^2}
\end{equation}

Following the Asimov approximation, the significance for individual bins is calculated as a Poisson ratio of likelihoods modified to incorporate  systematic uncertainties on the background  \cite{Cowan:2012}:  
%(\ref{eq:significance_wsyst})

\begin{equation}
\label{eq:significance_wsyst}
Z^\prime_{bin} = \sqrt{ 2 \cdot \bigg( (s+b) \ln[\frac{ (s+b) (b+\sigma_b^2) } {b^2 + (s+b) \sigma_b^2} ]- \frac{b^2}{\sigma_b^2} \ln[1 + \frac{\sigma_b^2 s}{b(b+\sigma_b^2)} ] \bigg) }
\end{equation}

This metric has the advantage that it accounts for background systematics and is still valid for $s$ not $\ll b$.  Expected significances are shown in Figure \ref{fig:expected_significance_monozll}, with regions the ATLAS and CMS experiments should be sensitive to, greater than 2, highlighted.

%\begin{equation}
%\label{eq:significance}
%Z = \sqrt{2 \bigg( (s+b) \ln{1 + \frac{s}{b}} - S \bigg) }
%\end{equation}


\paragraph{Conclusions} The Mono-Z(ll) provides experimental coverage of the pseudoscalar 2HDM model for a broad part of the parameter space. The light pseudoscalar $a$ can be probed up to mass values of $\approx\unit[350]{GeV}$, depending on the choice of parameters. The Mono-Z channel is sensitive mostly in the region of $\tanb<4$.

\begin{figure}
\centering
\includegraphics[width=0.8\textwidth]{texinputs/04_grid/figures/monoz/leptonic/xs_2d_inclusive_26300.pdf}
\includegraphics[width=0.8\textwidth]{texinputs/04_grid/figures/monoz/leptonic/tanbma_xsec_ll.pdf}
\caption{Inclusive cross-sections for $pp\rightarrow \lp\lm\chi\overline{\chi}$ in the \ma-\mA (top) and \ma-\tanb scans (bottom).}
\label{fig:monoz_ll_xs_inclusive}
\end{figure}


\begin{figure}
\centering
\includegraphics[width=0.6\textwidth]{texinputs/04_grid/figures/monoz/leptonic/SinThetaScan_xsecs.pdf}
\caption{For two different mass points, this figure shows the cross section $pp\rightarrow\chi\chi\ell\ell$ as a function of \sinp. 
For $\ma < 350$ GeV, $a$ decays solely to dark matter particles.  
As a consequence, the mixing angle only impacts the heavy scalar's branching fraction to $aZ$ and 
cross section strictly increases with \sinp.  
For \ma above 350 \GeV, \ttbar~ decays become accesible, introducing additional \sinp and \cosp dependences for the branching fraction 
of $a\rightarrow\chi\chi$.  For \ma above 350 GeV for large values of \sinp, there is a turnover point where the reduced $a\rightarrow\chi\chi$ branching fraction outweights the increased $H \rightarrow aZ$ branching and the net cross section decreases.}
\label{fig:monoz_ll_sinp_scan_xsec}
\end{figure}

\begin{figure}
\includegraphics[width=\textwidth]{texinputs/04_grid/figures/monoz/leptonic/acceptance_dmwg-final_26300.pdf}
\includegraphics[width=\textwidth]{texinputs/04_grid/figures/monoz/leptonic/xs_2d_dmwg-final_26300_yield40fb.pdf}
\caption{Acceptance and event yields in the  \ma-\mA plane after applying the final selection. Event yields assume an integrated luminosity of $40~\ifb$. The acceptance is maximal for $\mA > \ma$, where it reaches 50 \%. In the inverted mass region $\mA < \ma$, lower values of 10-30\% are observed. In the intermediate region around $\mA \approx \ma + \mZ$, the acceptance is strongly suppressed as the a and Z bosons are produced approximately at rest.}
\label{fig:monoz_ll_acceptance}
\end{figure}

\begin{figure}
\centering
\includegraphics[width=0.6\textwidth]{texinputs/04_grid/figures/monoz/leptonic/tanbma_ae_ll.pdf}
\caption{Acceptances across the \ma-\tanb scan.  Acceptance is flat over \tanb for constant values of \ma.}
\label{fig:monoz_ll_tanbma_acceptance}
\end{figure}

\begin{figure}
\centering
\includegraphics[width=0.6\textwidth]{texinputs/04_grid/figures/monoz/leptonic/tanbma_yield_ll.pdf}
\caption{Event yield in the \ma-tanBeta grid, for an integrated luminosity of $40~\ifb$.  The number of expected events diminshes with increasing tanBeta and \ma.  \mA fixed to 600 GeV and sinTheta to 0.35}
\end{figure}



\begin{figure}
\centering
%\includegraphics[width=0.9\textwidth]{texinputs/04_grid/figures/monoz/leptonic/mA_Scan.pdf}
\includegraphics[width=0.6\textwidth]{texinputs/04_grid/figures/monoz/leptonic/mAscan_for_ma200.pdf}
\caption{The position of the Jacobian peak in the \MET distribution depends on the values of \mH, \ma, and $M_{Z}$.  This figure shows a scan of \mH values for fixed \ma and \mA = \mH.  Increasing the difference between \mH and \ma shifts the location of the peak towards high energies, whereas for small mass splittings the \MET distribution is soft and most events will fail to pass the \MET selection criteria.}   
\label{fig:monoz_ll_mA_scan}
\end{figure}

\begin{figure}
\centering
\includegraphics[width=0.45\textwidth]{texinputs/04_grid/figures/monoz/leptonic/inclusive_h_mz_lep.pdf}
\includegraphics[width=0.45\textwidth]{texinputs/04_grid/figures/monoz/leptonic/inclusive_h_m_med_dm.pdf}
\caption{Distributions of the invariant mass of the dilepton (left) and $\chi\overline{\chi}$ systems (right) with no selection applied in addition to the generation cuts. The $M_{ll}$ distribution is centered around the Z boson mass independent of the chosen parameter point, indicating that there is no contribtion from $\gamma*$ exchange. The $M_{\chi\overline{\chi}}$ distribution }
\label{fig:monoz_kin_inclusive}
\end{figure}

\begin{figure}
\centering
\includegraphics[width=0.45\textwidth]{texinputs/04_grid/figures/monoz/leptonic/presel_h_balance.pdf}
\includegraphics[width=0.45\textwidth]{texinputs/04_grid/figures/monoz/leptonic/presel_h_dphi_met_ll.pdf}
\includegraphics[width=0.45\textwidth]{texinputs/04_grid/figures/monoz/leptonic/presel_h_dr_ll.pdf}
\caption{Distributions of the main selection variables after preselection: \pt balance (top panel), $\Delta\Phi$ (middle) and $\Delta R$ (bottom). The shown parameter points illustrate the different qualitative behavior in the three different mass regions. }
\label{fig:monoz_kin_presel}
\end{figure}


\begin{figure}
\centering
\includegraphics[width=0.48\textwidth]{texinputs/04_grid/figures/monoz/leptonic/TanbScan_mA600_ma150_MET.pdf} 
\includegraphics[width=0.48\textwidth]{texinputs/04_grid/figures/monoz/leptonic/SinpScan_mA600_ma250_MET.pdf} 
\caption{\MET distribution after preselection for scans of $\tan{\beta}$ for fixed \mA =600 \GeV and \ma = 150 \GeV (left) and for $\sin{\theta}$ for fixed \mA = 600 \GeV and \ma = 250 \GeV (right).  These parameter have little impact on the event's kinematic distributions, except for small values of \tanb where there is a slight softening and broadening of the \MET distribution caused by the increased contribution from the top box feynman diagram.}
\label{fig:monoz_kin_tanb_sintheta}
\end{figure}


\begin{figure}
\centering
\includegraphics[width=0.48\textwidth]{texinputs/04_grid/figures/monoz/leptonic/mDMScan_mA600_ma250_MET.pdf}
\includegraphics[width=0.47\textwidth]{texinputs/04_grid/figures/monoz/leptonic/mDMScan_ma250.pdf}
\caption{\MET distributions following preselection are shown (left) scaled to 40 \ifb and (right) normalized to unity are shown for different values of \mDM with fixed \mA = 600 \GeV and \ma = 250 \GeV.  In the $\mDM < \frac{\ma}{2}$ region, \mDM has no effect on event yield or \MET distribution, at $\mDM = \frac{\ma}{2}$ a resonant enhancement to the cross section occurs, and in the off-shell region where  $\mDM > \frac{\ma}{2}$ cross section steeply drops.  The \MET shape remains the same up to, and even slightly above, $\mDM = \frac{\ma}{2}$, but further off shell the \MET distribution becomes increasingly disperse.  For \mDM = 200 \GeV, DM can still decay on-shell through the $A$.  For \mDM = 500 \GeV both pseudoscalaras are off-shell leading to an event yield too low to fit on the figure on the left and a \MET distribution without structure.}
\label{fig:dm_scan_ll}
\end{figure}


\begin{figure}
\centering
\includegraphics[width=0.45\textwidth]{texinputs/04_grid/figures/monoz/leptonic/dmwg-final_h_pt_med_dm.pdf}
\includegraphics[width=0.45\textwidth]{texinputs/04_grid/figures/monoz/leptonic/dmwg-final_h_mt_total.pdf}
\caption{\MET and  MT distributions in the signal region. The \MET distribution shows a Jacobian structure in the $\mA > \ma$ regime, the location of which strongly depends on $\mA$. In the region of inverted mass hierarchy $\mA < \ma$, the spectrum is less structured and does not fall off as steeply towards higher values. For a small mass splitting of $\ma-\mA\approx M_{Z}$, the spectrum is shifted to much lower values of \MET. The MT distribution allows to access the resonant nature of the process. Clear mass peaks are present for the normal mass hierarchy. In the inverted region, the MT distribution is more sensitive to the mass difference $\ma-\mA$ than the \MET distribution, allowing to differentiate between signal hypotheses that give near-identical \MET distributions. }
\label{fig:monoz_kin_final}
\end{figure}


\begin{figure}
\centering
\includegraphics[width=0.6\textwidth]{texinputs/04_grid/figures/monoz/leptonic/mAma_Significance_ll.pdf}
\includegraphics[width=0.6\textwidth]{texinputs/04_grid/figures/monoz/leptonic/tanbma_Significance_ll.pdf}
\caption{Expected significances are calculated using published background estimates and assuming a reconstruction efficiency of 75\%.  The ATLAS and CMS experiments are expected to be sensitive to regions with significances greater than 2.}
\label{fig:expected_significance_monozll}
\end{figure}

