In the first part of this section we present the sensitivity estimates for the main signatures that are sensitive to the model, in the parameter scans chosen for comparison. These estimates are based on the reinterpretation of existing results that contain different amounts of public information, using generator-level simulation of the chosen grid points. In the case of the mono-Higgs results from~\cite{Aaboud:2017yqz}, model-independent limits are available. 

The second part of this section briefly outlines additional signatures that are also sensitive to this model. [to be completed]

Lastly, we show the projections for the sensitivity of searches of choice for the luminosity at the end of Run-2. [to be completed]

\subsection{Studies of the $\monohbb$ signature}
\label{sec:sensi_monohbb}
%What is commented out in the text below is in here. 
%The studies of the \monohbb channel presented here are based on MC simulations with version 2.4.3 of \mg~\cite{Alwall:2014hca} using a Universal FeynRules Output \cite{Degrande:2011ua} implementation of the 2HDM with a Yukawa sector of type II with DM mediator~(\hdma), as provided by the authors of \cite{Bauer:2017ota}. 
The NNPDF30\_lo\_as\_0130 set of parton distribution functions (PDF) at leading order in the five-flavor scheme, which assumes a massless $b$-quark, with $\alpha_{S}(m_{Z}) = 0.130$ is used for these simulations~\cite{Ball:2014uwa}. For consistency, five-flavor scheme and $m_b=0~\GeV$ are chosen for the matrix element (ME) computation in \mg.

The ME generated for the parton-level studies presented in the following is $ g g \to h  \chi \chi$ represented in  \autoref{fig:feyn_hdm}
%ref to feynmangraph, else cite Bauer:2017ota
The only exception is the $\ma-\tanb$ scan which will be discussed in the following and is summarised in \autoref{fig:monoHbb_sensi_full_ma_tanb}. In this scan also the ME $b b \to h  \chi \chi$ is generated because at high $\tanb$, the $b b$ initiated process can have an amplitude of a similar magnitude as the gluon fusion initiated process from~\autoref{fig:feyn_hdm}~\cite{Bauer:2017ota}. 
The gluon fusion is dominant in all the remaining parameter space, therefore the $bb$ initiated process and other negligible contributions are not considered explicitly for all the scans.

\subparagraph{Signal kinematics}

The free parameters of the \hdma model fall into two categories:
\begin{itemize}
\item those which only affect total signal cross section;
\item those which, in additional to the total cross section, also affect the kinematics, primarily the \MET distribution.
\end{itemize}
In the following, the free parameters of the \hdma model are studied in the context of the \monohbb signature with a particular focus on the latter category of parameters, as it emphasizes the kinematic diversity of potential new physics contributions represented in this simplified model.


The masses $\mA$ and $\ma$ of the pseudoscalars $A$ and $a$, which represent the two mediators in \autoref{fig:feyn_hdm}, affect the kinematics of the $\monohbb$ in a profound way by changing the location of the Jacobian peak in the \MET distribution. This effect is crucial to searches for \monohbb such as  \cite{Aaboud:2017yqz}, since the $\MET$ observable can be used to reduce many SM backgrounds, which are typically characterised by low  $\MET$, unlike DM signal processes with potentially very high $\MET$. In other words, the distribution of \met determines the sensitivity of the search.

The Jacobian peak is the result of a resonantly produced pseudoscalar $A$ decaying in the $2\to1\to2$ process $gg\to A\to a h$, 
where the Higgs boson proceeds to decay into a visible final state as $h\to b b$, and the light pseudoscalar into an invisible one as $a\to \chi\chi$. 
%The kinematics of  $1\to 2$ processes are fixed by the masses of the involved particles.
Thus, the resonant $A \to a h $  process has a sharply peaked resonance in the invariant mass distribution of the final state system with a  width determined by the widths of $a$, $A$, and $h$. This results in a peak in the momentum distribution of the DM system and in its transverse component that is reconstructed as $\MET$ in the detector.
%This peak in the $\MET$ distribution resulting from the resonant signal process  is called the Jacobian peak.

Since it is determined by the masses of the particles involved in the decay, 
the location of the Jacobian peak can be calculated analytically~\cite{Bauer:2017ota}:
\begin{align}
E^{\mathrm{miss},max}_T \approx \frac{\sqrt{\left(\mA^2 -\ma^2 -M_{h}^2\right)^2 - 4 \ma^2 M_{h}^2}}{2\mA}\,.
\label{eq:monoH_peak_met}
\end{align}
Thus, increasing $\mA$ results in  a Jacobian peak at higher $\MET$, as shown in \autoref{fig:monoHbb_mA_scan_met}.
Conversely, models with higher $\ma$ have a Jacobian peak at lower $\MET$, as indicated in \autoref{fig:monoHbb_ma_scan_met}. 

\begin{figure}[tbp]
\centering
\includegraphics[width=0.7\textwidth]{texinputs/04_grid/figures/monoHbb_m_large_A_scan_MET_liny_norm2one.pdf}
\caption[$\MET$ distribution in  \monohbb events for different $\mA$]
{
Missing transverse momentum distribution  \monohbb signal events at parton level for five representative models with different $\mA(=\mH=\mHc)$
and fixed $\ma = 300$ GeV, $ \sinp = 0.35, \tanb = 1, \mDM = 10$ GeV and $ \lap1 = \lap2 = \lam3 = 3 $. 
Models with a larger $\mA-\ma$ splitting have harder \met (cf.  \autoref{eq:monoH_peak_met}). 
%
% interpretation for the text
%This is due to the resonant Jacobian peak shifting to higher values.}
}
\label{fig:monoHbb_mA_scan_met}
\end{figure}


\begin{figure}[tbp]
\centering
\includegraphics[width=0.7\textwidth]{texinputs/04_grid/figures/monoHbb_m_small_a_scan_MET_liny_norm2one.pdf}
\caption[$\MET$ distribution in  \monohbb events for different $\ma$]
{
Missing transverse momentum distribution in  \monohbb signal events at parton level for four representative models with different $\ma$ 
and fixed $ \mA=\mH=\mHc= 700$ GeV, $ \sinp = 0.35, \tanb = 1, \mDM = 10$ GeV and $ \lap1 = \lap2 = \lam3 = 3 $. 
Models with higher $\ma$ have softer \met (cf. \autoref{eq:monoH_peak_met}).
%
% interpretation for the text
%This is due to the resonant Jacobian peak shifting to lower values.
}
\label{fig:monoHbb_ma_scan_met}
\end{figure}


In conclusion, the $\mA$ and $\ma$ parameters strongly affect the sensitivity of a search for the \hdma model using the \monohbb signature because they determine the location of the Jacobian peak in the \met distribution. Therefore, one of the proposed parameter scans for the \hdma model is in the ($\ma$,$\mA$) plane.


Some fraction of signal events is due to non-resonant $2 \to 3$ processes $gg \to h \chi \chi$ which is represented in \autoref{fig:feyn_hdm_box}. 
%link into thereory section graph or cite paper
Due to the larger number of kinematic degrees of freedom, the  invariant mass of the final state system is broadly distributed in these processes.
Consequently, this results in a broad and soft \met distribution that is clearly distinct from the Jacobian peak discussed above.
The models shown in \autoref{fig:monoHbb_mA_scan_met} and \autoref{fig:monoHbb_ma_scan_met} also have small contributions from non-resonant processes.


The mass of the heavy neutral scalar Higgs boson $H$ has an indirect effect on the rate and kinematics of the signal. 
This is caused by the dependence of the coupling strengths and thus decay widths of  the pseudoscalars $A$ and $a$ on  $\mH$~\cite{Bauer:2017ota}. 
Therefore, a change of $\mH$ can affect the relative contribution of resonant versus non-resonant signal processes, as illustrated in \autoref{fig:monoHbb_mH_scan_met}.  
The choice $\mH = \mA$ results in a detectable total cross section for many signal points and a dominant contribution of the resonant signal process, resulting in diverse experimental signatures as demonstrated in \autoref{fig:monoHbb_mA_scan_met} and \autoref{fig:monoHbb_ma_scan_met}. In addition, this choice results in about equal contributions to the sensitivity through the \monoz and \monoh signatures, highlighting their complementarity. Henceforth,  $\mH = \mA$ is adapted for all scans. For simplicity, the case of the neutral scalar $H^{\pm}$ being mass-degenerate to $H$ is considered in the following, as it does not affect the \hdma model kinematics in the \monohbb signature.

\begin{figure}[tbp]
\centering
\includegraphics[width=0.7\textwidth]{texinputs/04_grid/figures/monoHbb_mH_scan_MET_liny.pdf}
\caption[$\MET$ distribution in \monohbb events for different $\mH$]
{The \MET distribution of the production cross section of \monohbb signal events for five representative models with different $\mH = \mHc$ 
and fixed $ \mA=800$ GeV, $\ma = 300 $ GeV,  $ \sinp = 0.35, \tanb = 1, \mDM = 10$ GeV and $ \lap1 = \lap2 = \lam3 = 3 $. 
%
% interpretation for the text
%Models with $\mH = \mHc \geq \mA $ have a stronger resonant contribution, and a larger cross section. The resonant contribution is enhanced more strongly with higher $\mH = \mHc$. For $\mH = \mHc < \mA$ the resonant contribution is suppressed.
}
\label{fig:monoHbb_mH_scan_met}
\end{figure}



%%%%% text related to scan of sinp
The sine of the mixing angle between the two pseudoscalars $A$ and $a$, $\sinp$,
affects not only the cross section, but also the shape of the \MET\ distribution, as shown in \autoref{fig:monoHbb_sinp_scan_mA600_ma200_met}. 
For the resonant diagram $gg\rightarrow A \rightarrow ah \rightarrow \chi\bar{\chi}h$, 
the product of cross section times branching ratios 
${\cal B}(A\rightarrow ah){\cal B}(a \rightarrow \chi\bar{\chi})$ 
scales with $\sin^2\theta\cos^6\theta$, while for the diagram 
$gg\rightarrow a \rightarrow A^*h \rightarrow \chi\bar{\chi}h$, the 
product of cross section times branching ratios 
${\cal B}(a\rightarrow Ah){\cal B}(A \rightarrow \chi\bar{\chi})$ 
scales with $\sin^6\theta\cos^2\theta$. Therefore, at small \sinp, the resonant 
diagram $A\rightarrow ah$ is the dominant production mode and the \MET\ distribution 
has a Jacobian peak following \autoref{eq:monoH_peak_met}; while at large \sinp, the 
$a\rightarrow A^*h$ diagram starts to dominate and produces a second peak at a lower 
\MET\ value.


\begin{figure}[tbp]
\centering
\includegraphics[width=\textwidth]{texinputs/04_grid/figures/monoHbb_sinp_scan_MA600_Ma200_MET_liny_norm2one.pdf}
\caption[$\MET$ distribution in $h\rightarrow bb + \MET$ events for different 
$\sinp$ for $\mA = \mH = \mHc = 600 $ GeV and $\ma = 200$ GeV]
{
Missing transverse momentum distribution of $h\rightarrow bb + \MET$ signal 
events at parton level for five representative models with different $\sinp$ and
 fixed $\mA = \mH = \mHc = 600 $~GeV, $\ma = 200$~GeV, $ \mDM = 10$~GeV, $\tanb = 1$, 
and $ \lap1 = \lap2 = \lam3 = 3 $. 
The shape of the $\MET$ distribution does not change much  
for $\sinp < 0.7$, then changes significantly for $\sinp\geq 0.7$. 
When $\sinp=0.9$, the diagram $gg\rightarrow a\rightarrow A^*h \rightarrow \chi \bar{\chi} h$, 
producing a \MET peak at around 60~GeV, starts to dominate.
%
}
\label{fig:monoHbb_sinp_scan_mA600_ma200_met}
\end{figure}



%%% text related to tanbeta-ma scan
The shape of \MET\ distribution also has a non-trivial dependence on \tanb, as can be seen in \autoref{fig:monoHbb_tanb_scan_met}.
As discussed in the sensitivity study later, at small \tanb, the Yukawa coupling 
to top quark is large and the signal production mode is dominated by the 
non-resonant 3-body processes $gg\rightarrow h\chi\bar{\chi}$, which gives a broad 
and soft \MET\ spectrum. As \tanb increases, the contribution of 
resonant production increases as well and the Jacobian peak also appears.
When the pseudoscalar $A$ is produced off-shell, i.e. when $\mA<\ma+\mh$, the shapes 
of \MET\ distributions become similar and the dependence on \tanb disappears.


\begin{figure}[tbp]
\centering
\begin{subfigure}{0.48\textwidth}
\includegraphics[width = \textwidth]{texinputs/04_grid/figures/monoHbb_tanb_scan_MA600_Ma100_MET_liny_norm2one.pdf}
\end{subfigure}
~
\begin{subfigure}{0.48\textwidth}
\includegraphics[width = \textwidth]{texinputs/04_grid/figures/monoHbb_tanb_scan_MA600_Ma200_MET_liny_norm2one.pdf}
\end{subfigure}
\\
\centering
\begin{subfigure}{0.48\textwidth}
\includegraphics[width = \textwidth]{texinputs/04_grid/figures/monoHbb_tanb_scan_MA600_Ma350_MET_liny_norm2one.pdf}
\end{subfigure}
~
\begin{subfigure}{0.48\textwidth}
\includegraphics[width = \textwidth]{texinputs/04_grid/figures/monoHbb_tanb_scan_MA600_Ma500_MET_liny_norm2one.pdf}
\end{subfigure}
\caption[$\MET$ distribution in $h\rightarrow bb + \MET$ events for different 
$\tanb$ for $\mA = \mH = \mHc = 600 $ GeV]
{
Missing transverse momentum distribution of $h\rightarrow bb + \MET$ signal 
events at parton level with different $\tanb$ and
 fixed $\mA = \mH = \mHc = 600 $~GeV, $ \mDM = 10$~GeV, $\sinp = 0.35$, 
and $ \lap1 = \lap2 = \lam3 = 3 $. The values of $\ma$ are set to 100, 200, 
350, and 500~GeV, respectively.
The shapes of the $\MET$ distributions for different $\tanb$ are similar when 
$\mA < \mh+\ma$. Note, in these figures, both the contributions of $gg$ and $b\bar{b}$ 
initiated processes are included and a combined histogram is produced 
according to their corresponding cross sections.
}
\label{fig:monoHbb_tanb_scan_met}
\end{figure}



%\textcolor{red}{[CMS contribution]}
%The $\lambda$ parameters ...

The mass of the DM fermion $\mDM$ can change the total cross section and shape of the $\MET$ distribution,
depending on the mass hierarchy of the $A,a,h,\chi$ particles. This is demonstrated in \autoref{fig:monoHbb_mDM_scan_met}. 
Provided on-shell  decays $a\to\chi\chi$ are possible, i.e., $\mDM < \ma/2$, the exact value of \mDM has no effect on either kinematics or the total cross section. 
The only exception is the case $\ma/2 > \mDM > \frac{1}{2}(\ma - M_h)$. In this  $\mDM$ range, the non-resonant process $a \to h A^*\left(\chi\chi\right) $ is kinematically inaccessible. 
This reduces the overall cross section relative to the $\mDM \leq \frac{1}{2}(\ma - M_h)$ case, and slightly changes the soft part of the total $\MET$ spectrum. 
But since the contribution of the $a \to h A^*\left(\chi \chi\right)$ is minor in any case, the differences are negligible.

If the DM particle mass is exactly on threshold, i.e., $\mDM = \ma/2$, the total cross section is resonantly enhanced. 
This resonant threshold enhancement drops rapidly towards both higher and lower $\mDM$. Furthermore, the shape of the $\MET$ distribution at threshold, 
where amplitudes involving $a\to\chi\chi$ decays make up a larger fraction of the signal, differs significantly from the one below threshold. Below threshold ($\mDM > \ma/2$), the total cross section quickly drops by several orders of magnitude. In this regime, 
the shape of the $\MET$ distribution changes with $\mDM$ continuously.

\begin{figure}[tbp]
\centering
\includegraphics[width=0.7\textwidth]{texinputs/04_grid/figures/monoHbb_mDM_scan_MET_liny_norm2one.pdf}
\caption[$\MET$ distribution in \monohbb events for different $\mDM$]
{
Missing transverse momentum distribution  of \monohbb signal events at parton level for five representative models with different $\mDM$
and fixed $\mA = \mH = \mHc = 600 $ GeV $\ma = 250$ GeV, $ \sinp = 0.35, \tanb = 1$ and $ \lap1 = \lap2 = \lam3 = 3 $. 
The shape of the $\MET$ distribution does not change for $\mDM < \ma/2$, then changes significantly for $\mDM>=\ma/2$.
%
}
\label{fig:monoHbb_mDM_scan_met}
\end{figure}



\subparagraph{Sensitivity estimate}
\begin{figure}[tbp]
\centering
\begin{subfigure}{0.48\textwidth}
\includegraphics[width = \textwidth]{texinputs/04_grid/figures/monoHbb_parton_level_cross_section_bin_1_ma_vs_mA_lin.pdf}
\end{subfigure}
~
\begin{subfigure}{0.48\textwidth}
\includegraphics[width = \textwidth]{texinputs/04_grid/figures/monoHbb_parton_level_cross_section_bin_2_ma_vs_mA_lin.pdf}
\end{subfigure}
\\
\centering
\begin{subfigure}{0.48\textwidth}
\includegraphics[width = \textwidth]{texinputs/04_grid/figures/monoHbb_parton_level_cross_section_bin_3_ma_vs_mA_lin.pdf}
\end{subfigure}
~
\begin{subfigure}{0.48\textwidth}
\includegraphics[width = \textwidth]{texinputs/04_grid/figures/monoHbb_parton_level_cross_section_bin_4_ma_vs_mA_lin.pdf}
\end{subfigure}
\caption[$h\to bb + \MET$ cross section binned in $\MET$, $\mA$ - $\ma$ plane ]
{
The production cross section of $h\to bb + \MET$ signal events at parton level as a function of $(\mA,\ma)$ in each of the four \met bins. 
The remaining parameters take the values
$ \mH=\mHc= \mA, \sinp = 0.35, \tanb = 1, \mDM = 10$ GeV and $ \lap1 = \lap2 = \lam3 = 3 $.
}
\label{fig:monoHbb_xsec_bins_mA_ma}
\end{figure}

\begin{figure}[tbp]
\centering
\begin{subfigure}{0.48\textwidth}
\includegraphics[width = \textwidth]{texinputs/04_grid/figures/monoHbb_sensi_bin_1_ma_vs_mA_lin.pdf}
\end{subfigure}
~
\begin{subfigure}{0.48\textwidth}
\includegraphics[width = \textwidth]{texinputs/04_grid/figures/monoHbb_sensi_bin_2_ma_vs_mA_lin.pdf}
\end{subfigure}
\\
\centering
\begin{subfigure}{0.48\textwidth}
\includegraphics[width = \textwidth]{texinputs/04_grid/figures/monoHbb_sensi_bin_3_ma_vs_mA_lin.pdf}
\end{subfigure}
~
\begin{subfigure}{0.48\textwidth}
\includegraphics[width = \textwidth]{texinputs/04_grid/figures/monoHbb_sensi_bin_4_ma_vs_mA_lin.pdf}
\end{subfigure}
\caption[Sensitivity to the $h\to bb + \MET$ signal by $\MET$ bin, $\mA$ - $\ma$ plane]
{Estimated sensitivity to $h\to bb + \MET$ events as a function of $(\mA,\ma)$ in each of  the four \met bins. 
The sensitivity, defined in \autoref{eq:monoHbb_sensi}, is based on the limits with reduced model dependence from Ref.~\cite{Aaboud:2017yqz}. 
The remaining parameters take the values
$ \mH=\mHc= \mA, \sinp = 0.35, \tanb = 1, \mDM = 10$ GeV and $ \lap1 = \lap2 = \lam3 = 3 $.}
\label{fig:monoHbb_sensi_bins_mA_ma}
\end{figure}

\begin{figure}[tbp]
\centering
\includegraphics[width=0.7\textwidth]{texinputs/04_grid/figures/monoHbb_sensi_sum_bins_1_2_3_4_ma_vs_mA_lin.pdf}
\caption[Sensitivity to $h\to bb + \MET$ signals in $\mA$ - $\ma$ plane, summed across $\MET$ bins]
{
Sum over all $\MET$-bins of the estimated sensitivity to $h\to bb + \MET$ events as a function of $(\mA,\ma)$. 
The sensitivity, defined in \autoref{eq:monoHbb_sensi}, is based on the limits with reduced model dependence from Ref.~\cite{Aaboud:2017yqz}. 
The remaining parameters take the values $ \mH=\mHc= \mA, \sinp = 0.35, \tanb = 1, \mDM = 10$ GeV and $ \lap1 = \lap2 = \lam3 = 3 $.}
\label{fig:monoHbb_sensi_full_mA_ma}
\end{figure}


\begin{figure}[tbp]
\centering
\includegraphics[width=0.49\textwidth]{texinputs/04_grid/figures/monoHbb_sinp_scan_1_sensi_1D.pdf}
\includegraphics[width=0.49\textwidth]{texinputs/04_grid/figures/monoHbb_sinp_scan_2_sensi_1D.pdf}
\caption[Sensitivity to $h\to bb + \MET$ signals with different $\sinp$, summed across $\MET$ bins]
{
Sum over all $\MET$-bins of the estimated signal sensitivity to $h\to bb + \MET$ events as a function of the pseudoscalar mixing parameter $\sinp$, for $\ma = 200~\GeV$ and $\mH=\mHc=\mA = 600$~GeV~(left) as well as $\ma = 350~\GeV$ and $\mH=\mHc=\mA = 1000$~GeV~(right). The remaining parameters take the values
$\mDM = 10 $ GeV$, \tanb = 1,$ and $ \lap1 = \lap2 = \lam3 = 3 $.
The sensitivity, defined in \autoref{eq:monoHbb_sensi}, as well as the uncertainty on the sensitivity (shaded blue) 
 are based on the limits with reduced model dependence from Ref.~\cite{Aaboud:2017yqz} and the uncertainties described therein. 
}
\label{fig:monoHbb_sensi_full_sinp}
\end{figure}



\begin{figure}[tbp]
\centering
\includegraphics[width=0.7\textwidth]{texinputs/04_grid/figures/monoHbb_sensi_sum_bins_1_2_3_4_ma_vs_tanb_lin.pdf}
\caption[Sensitivity to $h\to bb + \MET$ signals in $\mA$ - $\tanb$ plane, summed across $\MET$ bins]
{
Sum over all $\MET$-bins of the estimated signal sensitivity to $h\to bb + \MET$ events as a function of $(\ma,\tanb)$. The sensitivity, defined in \autoref{eq:monoHbb_sensi}, is based on the limits with reduced model dependence from Ref.~\cite{Aaboud:2017yqz}. The remaining parameters take the values
$ \mH=\mHc=\mA = 600$ GeV, $ \sinp = 0.35, \mDM = 10$ GeV and $ \lap1 = \lap2 = \lam3 = 3 $.}
\label{fig:monoHbb_sensi_full_ma_tanb}
\end{figure}

\begin{figure}[tbp]
\centering
\includegraphics[width=0.7\textwidth]{texinputs/04_grid/figures/monoHbb_sensi_mDM_scan.pdf}
\caption[Sensitivity to $h\to bb + \MET$ signals with different $\mDM$, summed across $\MET$ bins]
{
Sum over all $\MET$-bins of the estimated signal sensitivity to $h\to bb + \MET$ events as a function of the DM mass $\mDM$. 
The sensitivity, defined in \autoref{eq:monoHbb_sensi}, as well as the uncertainty on the sensitivity (shaded blue)
are based on the limits with reduced model dependence from Ref.~\cite{Aaboud:2017yqz} and the uncertainties described therein. 
The remaining parameters take the values
$ \ma = 250 $ GeV$, \mH=\mHc=\mA = 600$ GeV, $ \sinp = 0.35, \tanb = 1,$ and $ \lap1 = \lap2 = \lam3 = 3 $. 
The sensitivity is constant below $\mDM < \ma/2$, and rapidly drops for $\mDM > \ma/2$. The sensitivity is resonantly enhanced for $\mDM = \ma/2$.}
\label{fig:monoHbb_sensi_full_mDM}
\end{figure}

The sensitivity estimate of ATLAS and CMS to the \hdma scenario through the \monohbb signature is based on limits on anomalous production of 125 GeV Higgs bosons in association with \met with minimal model dependence~\cite{Aaboud:2017yqz}. The limits are translated to parton level and compared to parton-level simulations of the \hdma scenario for the sensitivity estimate. This approach avoids the simulation of the detector response, which requires a significant amount of computing resources, and more iterations and refinements of the signal grid can be performed. 

The limits with minimal model dependence are provided in terms of the detector-level cross section of $\monohbb$ events $\sigma_{i}^{\mathrm{obs},\,\monohbb}$ as a function of \met in four bins $i=1,...4$~\cite{Aaboud:2017yqz}. 
To compare these values to the simulation results at parton level, 
an estimate of the detection efficiency $\varepsilon$ times the kinematic acceptance $\mathcal{A}$ of the event selections of the analysis is used for each of the four $\MET$ bins.
%This estimate is provided as one $(\mathcal{A\times\varepsilon})$ value for each of the four $\MET$ bins. 
Thus, the $(\mathcal{A}\times\varepsilon)_i$ figure represents the minimum probability
that an event generated at parton level in a given $\MET$ bin $i$ is reconstructed in that same $\MET$ bin and passes all analysis selections.
%The limits with minimal model dependence are provided separately for each of the four $\MET$ bins used in \cite{Aaboud:2017yqz}.
%Thus, the simulated evebts are binned into those bins (\autoref{fig:monoHbb_xsec_bins_mA_ma}).
Consequently, the cross section for \hdm production in the \hdma scenario at parton level $\sigma_{i}^{\mathrm{parton},\,\hdm}$ is calculated in the same $\MET$ bins as used in the \monohbb search. This starting point is shown in \autoref{fig:monoHbb_xsec_bins_mA_ma} using the scan in $(\mA,\ma)$ as a representative example. 
In the next step, the sensitivity $\sens_i$ for each of the \met bins $i=1,...4$ is calculated as
\begin{equation}
\label{eq:monoHbb_sensi_i}
\sens_i \equiv \frac{\sigma_{i}^{\mathrm{parton},\,\hdm} \times \mathcal{B}^{\mathrm{SM},\,h\to bb} \times (\mathcal{A\times\varepsilon})_{i} }
{\sigma_{i}^{\mathrm{obs},\,\monohbb}}\,,
\end{equation}
where $\mathcal{B}^{\mathrm{SM},\,h\to bb}$ is the $h\to bb$ branching ratio predicted by the SM for the 125~GeV Higgs boson. A representative example for this step is given in \autoref{fig:monoHbb_sensi_bins_mA_ma} for the scan in $(\mA,\ma)$.  A particular point in the $(\mA,\ma)$ parameter parameter space is excluded if $\sens_i \geq 1$. Finally, to obtain a single estimate for the total sensitivity $\senstot$ using all four $\MET$ bins, their individual contributions from \autoref{eq:monoHbb_sensi_i} are summed over\footnote{
This choice is made because the individual per-bin sensitivities follow a logarithmic metric, and because a model will typically populate several \met bins at a time. This implies that there could be models where $\sens_i<1$ in every bin, yet the sum from \autoref{eq:monoHbb_sensi} is $>1$.
Therefore, for a rigorous exclusion of a model based on the limits with minimal model dependence, the preferred approach would be to consider only the most sensitive bin for the exclusion.
}:
\begin{equation}
\label{eq:monoHbb_sensi}
\senstot \equiv \sum_{i\in\met~\mathrm{bins}} \sens_i\,.
\end{equation}
The resulting $\senstot$ is shown in \autoref{fig:monoHbb_sensi_full_mA_ma} for the example of the $(\mA,\ma)$ scan.


The scan of the sensitivity in the sense of \autoref{eq:monoHbb_sensi} in the $(\ma,\mA)$ plane is shown in \autoref{fig:monoHbb_sensi_full_mA_ma}.
The sensitivity decreases with increasing $\mA = \mH = \mHc$ for $\mA \geq 1$~TeV because the fraction of resonant signal events drops. 
This drop is caused by increasingly large $\Gamma_A$, 
which allows for an increasing fraction of non-resonant signal events, driven by events with very off-shell $A$. % ref ggF-> A -> ah feynman graph
%Non-resonant signal events have soft $\MET$ and thus the search is less sensitive to them, since the minimum accepted $\MET$ is $\MET \geq 150$ GeV.
Near the mass diagonal $\ma = \mA$, there is little to no sensitivity. 
This is because the Jacobian peak moves to low $\MET$ for a small mass splitting $|\mA - \ma|$
(\autoref{eq:monoH_peak_met}, \autoref{fig:monoHbb_mA_scan_met}, and \autoref{fig:monoHbb_ma_scan_met}).
Beyond this, the coupling $g_{Aah}$ is small when all Higgs bosons are nearly degenerate in mass, cf.~Equation~4.12 in Ref.~\cite{Bauer:2017ota}, %ref to eq.in theory part(?)
resulting in a small total cross section and therefore further decrease in sensitivity.
The sensitivity above the mass diagonal, $\mA > \ma$, is larger than below the mass diagonal.
Two parameter choices cause this asymmetry:
\begin{enumerate}
\item 
$\mA = \mH = \mHc$, i.e., the neutral and charged $CP$-even scalars have low masses below the diagonal, but high masses above it, introducing an asymmetry.
Another effect can be seen in \autoref{fig:monoHbb_mH_scan_met}: values of  $\mH = \mHc$ below the mass of the higher-mass pseudoscalar (in this case $A$)
give a reduced total cross section and a lower fraction of resonant signal events. Both effects reduce sensitivity;
\item 
$\sinp = 0.35 \neq 1/\sqrt{2}$, i.e. the mixing between the pseudoscalars $A$ and $a$ is asymmetric. 
$A$ couples more strongly to SM particles than $a$, and vice versa for the couplings to the DM fermion $\chi$.
So the situation below the diagonal corresponds to the case of $\sinp = \sqrt{1-0.35^2} \approx 0.938$ and $\mA > \ma$. 
As can be seen in \autoref{fig:monoHbb_sinp_scan_mA600_ma200_met},
this \sinp configuration has a higher fraction of non-resonant signal events with low \met, and correspondingly a lower sensitivity is found in \autoref{fig:monoHbb_sensi_full_sinp}.
\end{enumerate}

The scan of the sensitivity in the  $(\ma,\tanb)$ plane is shown in  \autoref{fig:monoHbb_sensi_full_ma_tanb}. 
At very low $\tanb$, the Yukawa coupling to top quarks is large, and most of the signal events come from non-resonant processes, as can be seen from \autoref{fig:monoHbb_tanb_scan_met}. % ref to tanb met scan
The non-resonant processes are characterised by soft $\MET$, which lowers the kinematic acceptance and reduces the sensitivity of the search.
For higher $\tanb$, the fraction of resonant events increases due to the reduced top Yukawa coupling, resulting in an increase of sensitivity.
However, reducing the top Yukawa coupling also reduces the total production cross section. 
This effect is sub-dominant below $\tanb \approx 1.2$, and the sensitivity increases with $\tanb$. 
But above $\tanb \approx 1.2$, the sensitivity loss due to reduced cross section outpaces the sensitivity gain due to a more resonant signal.
Overall, the search gets less sensitive with increasing $\tanb$ above $\tanb \approx 1.2$.
At very high $\tanb$ ($\geq 10$), this trend is reversed again because the $\tanb$ enhancement\footnote{The \hdma scenario assumes a Yukawa sector of type II.} of the 
coupling to $b$-quarks compensates for the small $b$-quark mass.
At this point $bb$ initiated processes start to dominate the production cross section and drive the increase in sensitivity.

The sensitivity to models with varying $\sinp$ is shown in \autoref{fig:monoHbb_sensi_full_sinp}.
The sensitivity vanishes at $\sinp=0$ and $\sinp=1$, since those values correspond to no mixing between $A$ and $a$, and thus no connection between the SM and the dark sector. 
For its intermediate values, the $\sinp$ parameter influences the couplings of the pseudoscalars to DM as well as to SM fermions, 
and also the coupling strength of trilinear scalar vertices such as $g_{Aah}$~\cite{Bauer:2017ota}. 
Increasing the couplings increases the total cross section. 
However, increasing some couplings can also increase $\Gamma_A$ and thereby decrease the resonant fraction of signal events and the sensitivity.
The upshot of this is that there can be more than one local maximum in the sensitivity curve, as shown the right panel of~\autoref{fig:monoHbb_sensi_full_sinp}. 
The precise dependence of the sensitivity on \sinp depends on the precise interplay of the couplings.
Because the couplings depend on all other model parameters including all the Higgs masses, 
tuning the $\sinp$ of a parameter scan to the sensitivity in a single point can lead to sub-optimal sensitivity in other points.

The sensitivity to models with varying $\mDM$ is shown in  \autoref{fig:monoHbb_sensi_full_mDM}.
Below the threshold of $\mDM < \ma/2$, the sensitivity is constant since the $\MET$ distribution and the total signal cross section remain invariant, as demonstrated in \autoref{fig:monoHbb_mDM_scan_met}.
At threshold, the sensitivity is enhanced because the partial width for $ a \to \chi \chi $ is enhanced, 
increasing the signal cross section.
Above threshold, the sensitivity drops rapidly because $\mDM > \ma/2$ requires an off-shell $a^{\star} \to \chi\chi$ decay, which is strongly suppressed by the typically  narrow width of $a$. 
The width of $a$ is substantially reduced once $a\to \chi \chi$ is kinematically inaccessible, 
as $\Gamma_{a\to \chi \chi}$ is a large contribution to the total width of $a$ for $\mDM \leq \ma/2$ \cite{Bauer:2017ota}.
There is a slight increase in sensitivity for $\mDM \approx \mA/2$ when the $A\to \chi\chi$ decay hits its kinematic threshold, yet the absolute sensitivity remains negligible.


The sensitivity estimates of the ATLAS and CMS \monohbb searches to the \hdma scenarios are based on limits on the minimally model-dependent anomalous production of 125 GeV Higgs bosons in association with \met in~\cite{Aaboud:2017yqz}. 
As these limits are set in terms of the observed production cross-section of non-SM events with large $\MET$ and a Higgs boson, they can be compared directly to the cross-sections obtained from the \hdma model after folding the detection efficiency $\varepsilon$ times the kinematic acceptance $\mathcal{A}$ of the event selection. This approach reduces the need for computing resources to simulate further event generation steps and detector response. 
The variable of interest for the sensitivity study of the \monohbb searches is the ratio between the parton-level cross-section $\sigma_{i}^{\mathrm{parton},\,\hdm}$ times the $H\rightarrow b\bar{b}$ branching ratio $\mathcal{B}^{\mathrm{SM},\,h\to bb}$ predicted by the SM for the 125~GeV Higgs boson, multiplied by the acceptance $\mathcal{A}$ and detector efficiency $\varepsilon$, and the upper observed cross-section of the anomalous production of Higgs bosons in association with \met ($\sigma_{i}^{\mathrm{obs},\,\monohbb}$):

\begin{equation}
\label{eq:monoHbb_sensi}
\sens_i \equiv \frac{\sigma_{i}^{\mathrm{parton},\,\hdm} \times \mathcal{B}^{\mathrm{SM},\,h\to bb} \times (\mathcal{A\times\varepsilon})_{i} }
{\sigma_{i}^{\mathrm{obs},\,\monohbb}}\,,
\end{equation}

where  is the $h\to bb$ branching ratio predicted by the SM for the 125~GeV Higgs boson. This quantity is summed over the $i$ \met bins of the search, since the model will populate more than one \met bin at a time. A particular point in the space is excluded by the current search if $\sens_i \geq 1$. 

\begin{figure}[tbp]
\centering
\includegraphics[width=0.7\textwidth]{texinputs/04_grid/figures/monoHbb_sensi_sum_bins_1_2_3_4_ma_vs_mA_lin.pdf}
\caption[Sensitivity to $h\to bb + \MET$ signals in $\mA$ - $\ma$ plane, summed across $\MET$ bins]
{
Sum over all $\MET$-bins of the estimated sensitivity to $h\to bb + \MET$ events as a function of $(\mA,\ma)$. 
The sensitivity, defined as the sum of \autoref{eq:monoHbb_sensi} over the \met bins, is based on the limits with reduced model dependence from Ref.~\cite{Aaboud:2017yqz}. 
The remaining parameters take the values $ \mH=\mHc= \mA, \sinp = 0.35, \tanb = 1, \mDM = 10$ GeV and $ \lap1 = \lap2 = \lam3 = 3 $.}
\label{fig:monoHbb_sensi_full_mA_ma}
\end{figure}

The expected sensitivity of \monohbb searches to the \hdma model in the $(\ma,\mA)$ plane is shown in \autoref{fig:monoHbb_sensi_full_mA_ma}.
The sensitivity decreases with increasing $\mA = \mH = \mHc$ for $\mA \geq 1$~TeV because the fraction of resonant signal events drops. 
This drop is caused by increasingly large $\Gamma_A$, which allows for an increasing fraction of non-resonant signal events, driven by events with very off-shell $A$. 
%TODO: ref ggF-> A -> ah feynman graph
%Non-resonant signal events have soft $\MET$ and thus the search is less sensitive to them, since the minimum accepted $\MET$ is $\MET \geq 150$ GeV.
Near the mass diagonal $\ma = \mA$, there is little to no sensitivity. 
This is because the Jacobian peak moves to low $\MET$ for a small mass splitting $|\mA - \ma|$  (as shown in \autoref{eq:monoH_peak_met}, \autoref{fig:monoHbb_mA_scan_met}, and \autoref{fig:monoHbb_ma_scan_met}). 
Moreover, the coupling $g_{Aah}$ is small when all Higgs bosons are nearly degenerate in mass, cf.~Equation~4.12 in Ref.~\cite{Bauer:2017ota}, %ref to eq.in theory part(?)
resulting in a small total cross section and therefore a further decrease in sensitivity.
The sensitivity above the mass diagonal, $\mA > \ma$, is larger than below the mass diagonal.
Two parameter choices cause this asymmetry:
\begin{enumerate}
%CD: this said \mA = \mH = \mHc but the plot shows that we vary mH and fix mA
%CD I am not sure I understand this sentence? 
\item 
The choice of $\mA = \mH = \mHc$ forces the neutral and charged $CP$-even scalars to have lower masses below the diagonal and higher masses above the diagonal, leading to configurations with a lower fraction of resonant signal events. One can use \autoref{fig:monoHbb_mH_scan_met} to exemplify this behaviour. Considering the symmetry between the two pseudoscalars, when the neutral and charged scalars $\mH = \mHc$ are lighter than one of the two pseudoscalars (marked as $A$ in the figure, but effectively representing $a$ in the case of this scan because of the symmetry), one can see that non-resonant configurations are preferred. This also yields a reduced total cross-section. 
\item 
The choice of $\sinp = 0.35 \neq 1/\sqrt{2}$ means that the mixing between the pseudoscalars $A$ and $a$ is asymmetric. 
$A$ couples more strongly to SM particles than $a$, while the opposite happens to the DM fermion $\chi$.
The situation below the diagonal corresponds to the case of $\sinp = \sqrt{1-0.35^2} \approx 0.938$ and $\mA > \ma$. 
As it can be seen in \autoref{fig:monoHbb_sinp_scan_mA600_ma200_met}, this \sinp configuration yields a higher fraction of non-resonant signal events with low \met, and correspondingly a lower sensitivity is found, as also seen in \autoref{fig:monoHbb_sensi_full_sinp}.
\end{enumerate}


\begin{figure}[tbp]
\centering
\includegraphics[width=0.7\textwidth]{texinputs/04_grid/figures/monoHbb_sensi_sum_bins_1_2_3_4_ma_vs_tanb_lin.pdf}
\caption[Sensitivity to $h\to bb + \MET$ signals in $\mA$ - $\tanb$ plane, summed across $\MET$ bins]
{
Sum over all $\MET$-bins of the estimated signal sensitivity to $h\to bb + \MET$ events as a function of $(\ma,\tanb)$. 
The sensitivity, defined as the sum of \autoref{eq:monoHbb_sensi} over the \met bins, is based on the limits with reduced model dependence from Ref.~\cite{Aaboud:2017yqz}. 
The remaining parameters take the values $ \mH=\mHc=\mA = 600$ GeV, $ \sinp = 0.35, \mDM = 10$ GeV and $ \lap1 = \lap2 = \lam3 = 3 $.}
\label{fig:monoHbb_sensi_full_ma_tanb}
\end{figure}

The scan of the sensitivity in the  $(\ma,\tanb)$ plane is shown in  \autoref{fig:monoHbb_sensi_full_ma_tanb}. 
At very low $\tanb$, the Yukawa coupling to top quarks is large, and most of the signal events come from non-resonant processes, as can be seen from \autoref{fig:monoHbb_tanb_scan_met}. % ref to tanb met scan
The non-resonant processes are characterised by soft $\MET$, which lowers the kinematic acceptance and reduces the sensitivity of the search.
For higher $\tanb$, the fraction of resonant events increases due to the reduced top Yukawa coupling, resulting in an increase of sensitivity.
However, reducing the top Yukawa coupling also reduces the total production cross section. 
%This effect is sub-dominant below $\tanb \approx 1.2$, and the sensitivity increases with $\tanb$. 
Above $\tanb \approx 1.2$, the sensitivity loss due to reduced cross section outpaces the sensitivity gain due the resonant signal.
%Overall, the search gets less sensitive with increasing $\tanb$ above $\tanb \approx 1.2$.
At very high $\tanb$ ($\geq 10$), this trend is reversed again because the $\tanb$ enhancement\footnote{The \hdma scenario assumes a Yukawa sector of type II.} of the coupling to $b$-quarks compensates for the small $b$-quark mass.
At this point $bb$ initiated processes start to dominate the production cross section and drive the increase in sensitivity.


\begin{figure}[tbp]
\centering
\includegraphics[width=0.49\textwidth]{texinputs/04_grid/figures/monoHbb_sinp_scan_1_sensi_1D.pdf}
\includegraphics[width=0.49\textwidth]{texinputs/04_grid/figures/monoHbb_sinp_scan_2_sensi_1D.pdf}
\caption[Sensitivity to $h\to bb + \MET$ signals with different $\sinp$, summed across $\MET$ bins]
{
Sum over all $\MET$-bins of the estimated signal sensitivity to $h\to bb + \MET$ events as a function of the pseudoscalar mixing parameter $\sinp$, for $\ma = 200~\GeV$ and $\mH=\mHc=\mA = 600$~GeV~(left) as well as $\ma = 350~\GeV$ and $\mH=\mHc=\mA = 1000$~GeV~(right). The remaining parameters take the values
$\mDM = 10 $ GeV$, \tanb = 1,$ and $ \lap1 = \lap2 = \lam3 = 3 $.
The sensitivity, defined as the sum of \autoref{eq:monoHbb_sensi} over the \met bins, as well as the uncertainty on the sensitivity (shaded blue) 
 are based on the limits with reduced model dependence from Ref.~\cite{Aaboud:2017yqz} and the uncertainties described therein. 
}
\label{fig:monoHbb_sensi_full_sinp}
\end{figure}

The sensitivity as a function of $\sinp$ is shown in \autoref{fig:monoHbb_sensi_full_sinp}.
The sensitivity vanishes at $\sinp=0$ and $\sinp=1$, since those values correspond to no mixing between $A$ and $a$, and thus no connection between the SM and the dark sector. 
For its intermediate values, the $\sinp$ parameter influences the couplings of the pseudoscalars to DM as well as to SM fermions, as well as the coupling strength of the trilinear scalar vertices such as $g_{Aah}$~\cite{Bauer:2017ota}. 
Increasing these couplings increases the total cross section, but it can also increase $\Gamma_A$ and thereby decrease the resonant fraction of signal events and the overall search sensitivity.
For this reason, the dependence of the sensitivity on \sinp depends on the interplay of the couplings.
As a consequence, the sensitivity curve as a function of \sinp has more than one local maximum, as shown the right panel of~\autoref{fig:monoHbb_sensi_full_sinp}. 
%Because the couplings depend on all other model parameters including all the Higgs masses, tuning the $\sinp$ of a parameter scan to the sensitivity in a single point can lead to sub-optimal sensitivity in other points.

\begin{figure}[tbp]
\centering
\includegraphics[width=0.7\textwidth]{texinputs/04_grid/figures/monoHbb_sensi_mDM_scan_red.pdf}
\caption[Sensitivity to $h\to bb + \MET$ signals with different $\mDM$, summed across $\MET$ bins]
{
Sum over all $\MET$-bins of the estimated signal sensitivity to $h\to bb + \MET$ events as a function of the DM mass $\mDM$. 
The sensitivity, defined as the sum of \autoref{eq:monoHbb_sensi} over the \met bins, as well as the uncertainty on the sensitivity (shaded blue) are based on the limits with reduced model dependence from Ref.~\cite{Aaboud:2017yqz} and the uncertainties described therein. 
The remaining parameters take the values $ \ma = 250 $ GeV$, \mH=\mHc=\mA = 600$ GeV, $ \sinp = 0.35, \tanb = 1,$ and $ \lap1 = \lap2 = \lam3 = 3 $. 
The sensitivity is constant below $\mDM < \ma/2$, and rapidly drops for $\mDM > \ma/2$. The sensitivity is resonantly enhanced for $\mDM = \ma/2$.}
\label{fig:monoHbb_sensi_full_mDM}
\end{figure}

The sensitivity to models with varying $\mDM$ is shown in  \autoref{fig:monoHbb_sensi_full_mDM}.
Below the threshold of $\mDM < \ma/2$, the sensitivity is constant since the $\MET$ distribution and the total signal cross section remain unchanged.%, as demonstrated in \autoref{fig:monoHbb_mDM_scan_met}.
The region at threshold $\mDM = \ma/2 \pm 5 GeV$ (shaded in red in \autoref{fig:monoHbb_sensi_full_mDM}) is numerically unstable and should be avoided. 
%At threshold, the sensitivity is enhanced because the partial width for $ a \to \chi \chi $ is enhanced, increasing the signal cross section. However, this region is numerically unstable, 
%in the case i studied, staying only 1 GeV away from m_a = 2 * m_dm produced already results that looked perfectly fine
%in practice i think it is easy to spot when things go wrong. one just has to look at the sensitivity as a function of m_DM. for instance, this plot is a good example
%https://github.com/LHC-DMWG/DMWG-2HDM-whitepaper/blob/master/texinputs/04_grid/figures/monoHbb_sensi_mDM_scan.pdf
%the sensitivity should decrease with increasing m_DM. if the sensitivity is enhanced like it is the case at hand for m_DM values around 125 GeV then one should disregard the corresponding points because the enhancement of the sensitivity is due to a numerical artefact  
Above threshold, the sensitivity drops rapidly because $\mDM > \ma/2$ requires an off-shell $a^{\star} \to \chi\chi$ decay, which is strongly suppressed by the typically narrow width of $a$. 
The width of $a$ is substantially reduced once $a\to \chi \chi$ is kinematically inaccessible, as $\Gamma_{a\to \chi \chi}$ is a large contribution to the total width of $a$ for $\mDM \leq \ma/2$ \cite{Bauer:2017ota}.
There is a slight increase in sensitivity for $\mDM \approx \mA/2$ when the $A\to \chi\chi$ decay hits its kinematic threshold, yet the absolute sensitivity remains negligible.

\FloatBarrier

\subsection{Studies of the Z+\MET signature}

In the absence of generic limits on anomalous production of Z+\MET events, the expected sensitivity of the Z+\MET  searches to this model is approximated comparing the number of generator-level signal events to the respective background estimates. 

For the leptonic channel, the published background estimates corresponding to 36~fb$^{-1}$ of 13~TeV data~\cite{Aaboud:2017bja} are used, and a reconstruction efficiency of 75\% is assumed for signal events. 
The same selection cuts applied to data in ~\cite{Aaboud:2017bja} are applied to signal. 
Signal and background are binned in the same published \MET bins, and a conservative background systematic uncertainty of 20\% is assumed for $\MET < 120$ \GeV and 10\% for $\MET > 120$ \GeV. 

For the hadronic channel, $Z \to \nu\nu$ events in association with jets are simulated using Sherpa~2.2.1~\cite{Gleisberg:2008ta} and the matrix elements are calculated up to 2 partons at next-to-leading order and up to 4 partons at leading order. Signal and background are scaled to 40~fb$^{-1}$ of 13~TeV data.
The $Z (\to \nu\nu)$+jets events are analyzed at particle level with the same criteria used for the signal. 
The number of $Z \to \nu\nu$ events after applying the cuts is increased by a factor 2 to conservatively account for the contribution from other backgrounds. 
This factor is based on the ATLAS dark matter search in the mono-$Z$ hadronic signature using 3.2~fb$^{-1}$ of 13~TeV data~\ref{Aaboud:2016qgg}. 

Following the Asimov approximation, the significance for individual bins is calculated as a Poisson ratio of likelihoods modified to incorporate systematic uncertainties on the background \cite{Cowan:2012}:  
%(\autoref{eq:significance_wsyst})

\begin{equation}
\label{eq:significance_wsyst}
Z^\prime_{bin} = \sqrt{ 2 \cdot \bigg( (s+b) \ln[\frac{ (s+b) (b+\sigma_b^2) } {b^2 + (s+b) \sigma_b^2} ]- \frac{b^2}{\sigma_b^2} \ln[1 + \frac{\sigma_b^2 s}{b(b+\sigma_b^2)} ] \bigg) }
\end{equation}

This metric has the advantage that it accounts for background systematics and is still valid for $s >> b$.  
Similarly to the \monohbb, case, the total significance is defined as the per bin significances summed in quadrature.
 he ATLAS and CMS experiments are expected to be sensitive to regions with significances greater than 2.
 
The expected sensitivity of Z+\MET searches to the \hdma model in the $(\ma,\mA)$ plane is shown in \autoref{fig:expected_significance_monozll_mamA} for the leptonic case, and \autoref{fig:expected_significance_monozhad_mamA} for the hadronic case.

\begin{figure}
\centering
\includegraphics[width=0.6\textwidth]{texinputs/04_grid/figures/monoz/leptonic/mAma_Significance_ll.pdf}
\caption{Expected significances for the Z+\MET leptonic signature in the $(\ma,\mA)$ plane.} 
\label{fig:expected_significance_monozll_mamA}
\end{figure}

\begin{figure}
\centering
\includegraphics[width=0.6\textwidth]{texinputs/04_grid/figures/monoz/hadronic/grid_mA_ma_sum_bin100_sign_type3_bkg_uncert_0p10.pdf}
\caption{Expected significances for the Z+\MET hadronic signature in the $(\ma,\mA)$ plane, combining both boosted and resolved analyses.} 
\label{fig:expected_significance_monozhad_mamA}
\end{figure}

The sensitivity for the Z+\MET signatures in the $(\ma,\tanb)$ plane is shown in \autoref{fig:expected_significance_monozll_mamA} 

\begin{figure}
\centering
\includegraphics[width=0.6\textwidth]{texinputs/04_grid/figures/monoz/leptonic/tanbma_Significance_ll.pdf}
\caption{Expected significances for the Z+\MET leptonic signature in the $(\ma,\tanb)$ plane.} 
\label{fig:expected_significance_monozll_mamA}
\end{figure}

\begin{figure}
\centering
\includegraphics[width=0.6\textwidth]{texinputs/04_grid/figures/monoz/hadronic/grid_tanb_ma_sum_bin100_sign_type3_bkg_uncert_0p10.pdf}
\caption{Expected significances for the Z+\MET hadronic signature in the $(\ma,\tanb)$ plane, combining both boosted and resolved analyses.} 
\label{fig:expected_significance_monozhad_mamA}
\end{figure}

The leptonic Z+\MET search provides experimental coverage of this model for a broad part of the parameter space. The light pseudoscalar $a$ can be probed up to mass values of $\approx\unit[350]{GeV}$, depending on the choice of parameters. Leptonic Z+\MET searches are mostly sensitive in the region of $\tanb<4$. The hadronic Z+\MET search covers a smaller parameter space with respect to the leptonic search, but it is nevertheless complementary 

\FloatBarrier

\subsection{Studies of DM+heavy flavor signature}
%%%% \documentclass[11pt]{article}
%%% \usepackage{graphicx,epsfig}
%
%%% % Shorthand for \phantom to use in tables
%%% \newcommand{\pho}{\phantom{0}}
%%% \newcommand{\BibTeX}{\textsc{Bib\TeX}}
%%% \newcommand{\File}[1]{\texttt{#1}\xspace}
%%% \newcommand{\Macro}[1]{\texttt{\textbackslash #1}\xspace}
%%% \newcommand{\Option}[1]{\textsf{#1}\xspace}
%%% \newcommand{\Package}[1]{\texttt{#1}\xspace}
%%% \newcommand{\MADGRAPH}{\textsc{MadGraph}}
%%% \newcommand{\PYTHIA}{\textsc{Pythia}}
%
%%% \usepackage{subcaption}
%%% \usepackage{siunitx}
%%% \newcommand{\ttbar}{\ensuremath{\bar{t}t}}
%%% \newcommand{\bbbar}{\ensuremath{\bar{b}b}}
%%% \newcommand*{\TeV}{\ensuremath{\text{Te\kern -0.1em V}}}
%%% \newcommand*{\GeV}{\ensuremath{\text{Ge\kern -0.1em V}}}
%%% \newcommand{\pt}{\ensuremath{p_{\mathrm T}}}
%%% \newcommand{\met}{\ensuremath{E_{\mathrm T}^{\mathrm miss}}}
%%% \usepackage{biblatex}
%%% \bibliography{draft}
%
%%% \usepackage{lineno}
%%% \linenumbers
%
%
%
%%% \begin{document}
%
%
%In the following sections, the most important signatures involving either visible or invisible decays of the heavy Higgs bosons are reviewed. 
%
%
%%%%%-----------------------------------------------------------------------------------------
%%%%%-----------------------------------------------------------------------------------------
%\subsubsection{Scanning the parameter space}
%
%
%\paragraph{Scan of $\mathrm{M_{a}}$ and $\mathrm{M_{A}}$:}
%
%While the relevant kinematic distributions display no dependence on the aforementioned mixing angles, the same does not hold true for the masses, 
%
%The masses \ma and \mA influence the kinematics in the $t\bar{t}$ + \MET signature as well. As shown in \autoref{fig:kin_Ma}, the $E_{T}^{miss}$, and leading and trailing top quark $p_{T}$ distributions broaden with increasing $\mathrm{M_{a}}$. Similarly, for values of $\mathrm{M_{A}} < \mathrm{M_{a}}$, as $\mathrm{M_{A}}$ increases, the kinematic distributions mentioned above also broaden, as shown in \autoref{fig:kin_MA}.
%
%\begin{figure}
%  \centering
%  \begin{subfigure}[b]{0.49\textwidth}
%    \includegraphics[width=\textwidth]{texinputs/04_grid/figures/DMHF/benchmarking/MDM_1_MA_750_sinp_0.7071_tanb_1.0_SCAN_Ma/metlog.pdf}
%    \caption{$E_{T}^{miss}$}
%  \end{subfigure}
%  \begin{subfigure}[b]{0.49\textwidth}
%    \includegraphics[width=\textwidth]{texinputs/04_grid/figures/DMHF/benchmarking/MDM_1_MA_750_sinp_0.7071_tanb_1.0_SCAN_Ma/top1ptlog.pdf}
%    \caption{Leading top $p_{T}$}
%  \end{subfigure} \\
%  \begin{subfigure}[b]{0.49\textwidth}
%    \includegraphics[width=\textwidth]{texinputs/04_grid/figures/DMHF/benchmarking/MDM_1_MA_750_sinp_0.7071_tanb_1.0_SCAN_Ma/top2ptlog.pdf}
%    \caption{Trailing top $p_{T}$}
%  \end{subfigure}
%  \caption{The $E_{T}^{miss}$, leading and trailing top $p_{T}$ distributions for inclusive $t\bar{t}+\chi\bar{\chi}$ production for various values of $\mathrm{M_a}$, with $\mathrm{M_A}=750$ GeV, $\mathrm{M_H}=\mathrm{M_{H^{\pm}}}=750$ GeV, $\tan\beta=1$, and $\sin\theta=0.7071$.}
%  \label{fig:kin_Ma}
%\end{figure}
%
%\begin{figure}
%  \centering
%  \begin{subfigure}[b]{0.49\textwidth}
%    \includegraphics[width=\textwidth]{texinputs/04_grid/figures/DMHF/benchmarking/MDM_1_Ma_700_sinp_0.7071_tanb_1.0_SCAN_MA/metlog.pdf}
%    \caption{$E_{T}^{miss}$}
%  \end{subfigure}
%  \begin{subfigure}[b]{0.49\textwidth}
%    \includegraphics[width=\textwidth]{texinputs/04_grid/figures/DMHF/benchmarking/MDM_1_Ma_700_sinp_0.7071_tanb_1.0_SCAN_MA/top1ptlog.pdf}
%    \caption{Leading top $p_{T}$}
%  \end{subfigure} \\
%  \begin{subfigure}[b]{0.49\textwidth}
%    \includegraphics[width=\textwidth]{texinputs/04_grid/figures/DMHF/benchmarking/MDM_1_Ma_700_sinp_0.7071_tanb_1.0_SCAN_MA/top2ptlog.pdf}
%    \caption{Trailing top $p_{T}$}
%  \end{subfigure}
%  \caption{The $E_{T}^{miss}$, leading and trailing top $p_{T}$ distributions for inclusive $t\bar{t}+\chi\bar{\chi}$ production for various values of $\mathrm{M_A}$, with $\mathrm{M_a}=700$ GeV, $\mathrm{M_H}=\mathrm{M_{H^{\pm}}}=750$ GeV, $\tan\beta=1$, and $\sin\theta=0.7071$.}
%  \label{fig:kin_MA}
%\end{figure}
%%%%%-----------------------------------------------------------------------------------------
%\subsubsection{Comparison with DMsimp Pseudoscalar Model}
%
%To date, simplified models of DM (\texttt{DMsimp}) are used to interpret Run II CMS and ATLAS HF+DM searches. A comparison of the pertinent kinematic distributions between the pseudoscalar simplified model and the 2HDM+a model for the same value of $\mathrm{M_{a}}$ are shown in Fig.~\ref{kin_DMSimpV2HDMa}. The kinematics of the pseudoscalar \texttt{DMsimp} model with $\mathrm{M_{a}}=100$ GeV map directly onto those of the 2HDM+a model with $\mathrm{M_{a}}=100$ GeV, $\mathrm{M_{A}}=600$ GeV, $\mathrm{M_{H}}=\mathrm{M_{H^{\pm}}}=600$ GeV, $\sin\theta=0.7071$, and $\tan\beta=1$. From the mass distribution of the $\chi\bar{\chi}$ shown in Fig.~\ref{fig:mchichi_DMsimpV2HDMa}, it is evident that the 2HDM+a model contains contributions from both the light and heavy pseudoscalar mediator as in the \texttt{DMsimp} model.
%
%\begin{figure}
%  \centering
%  \begin{subfigure}[b]{0.49\textwidth}
%    \includegraphics[width=\textwidth]{texinputs/04_grid/figures/DMHF/benchmarking/MDM_1_Ma_100_MA_600_sinp_0.7071_tanb_1.0_VS_DMSimp_100_600_Decayed/metlog.pdf}
%    \caption{$E_{T}^{miss}$}
%  \end{subfigure}
%  \begin{subfigure}[b]{0.49\textwidth}
%    \includegraphics[width=\textwidth]{texinputs/04_grid/figures/DMHF/benchmarking/MDM_1_Ma_100_MA_600_sinp_0.7071_tanb_1.0_VS_DMSimp_100_600_Decayed/top1ptlog.pdf}
%    \caption{Leading top $p_{T}$}
%  \end{subfigure} \\
%  \begin{subfigure}[b]{0.49\textwidth}
%    \includegraphics[width=\textwidth]{texinputs/04_grid/figures/DMHF/benchmarking/MDM_1_Ma_100_MA_600_sinp_0.7071_tanb_1.0_VS_DMSimp_100_600_Decayed/top2ptlog.pdf}
%    \caption{Trailing top $p_{T}$}
%  \end{subfigure}
%  \caption{The $E_{T}^{miss}$, leading and trailing top $p_{T}$ distributions for inclusive $t\bar{t}+\chi\bar{\chi}$ production for various values of $\mathrm{M_A}$, with $\mathrm{M_a}=700$ GeV, $\mathrm{M_H}=\mathrm{M_{H^{\pm}}}=750$ GeV, $\tan\beta=1$, and $\sin\theta=0.7071$.}
%  \label{fig:kin_DMSimpV2HDMa}
%\end{figure}
%
%\begin{figure}
%  \centering
%  \includegraphics[width=0.6\textwidth]{texinputs/04_grid/figures/DMHF/benchmarking/MDM_1_Ma_100_MA_600_sinp_0.7071_tanb_1.0_VS_DMSimp_100_600_Decayed/mchichi.pdf}
%  \caption{The mass distribution of the $\chi\bar{\chi}$ system for \texttt{DMsimp} pseudoscalar models with $\mathrm{M_a}=100$ GeV and $\mathrm{M_a}=600$ GeV, compared with 2HDM+a with $\mathrm{M_a}=100$ GeV, $\mathrm{M_A}=600$ GeV, $\mathrm{M_H}=\mathrm{M_{H^{\pm}}}=600$ GeV, $\sin\theta=0.7071$ and $\tan\beta=1$.}
%  \label{fig:mchichi_DMsimpV2HDMa}
%\end{figure}
%
%In Fig.~\ref{fig:DMSimpV2HDMa}, relevant kinematic distributions, commonly employed in HF+DM searches, are mapped from the \texttt{DMsimp} pseudoscalar models to the 2HDM+a model, with the mediator masses corresponding to the additional light pseudoscalar in the latter model. The dashed distributions represent the \texttt{DMsimp} model, while the solid are the 2HDM+a model distributions. The $t\bar{t}+\chi\bar{\chi}$ process was generated at LO precision using both models. As can be seen, the kinematics do not change appreciably between the models generated at the same value of $\mathrm{M_{a}}$. A discussion on cross-section rescaling procedures can be found in the following section.
%
%\begin{figure}
%  \centering    
%  \begin{subfigure}[b]{0.49\textwidth}
%    \includegraphics[width=\textwidth]{texinputs/04_grid/figures/DMHF/benchmarking/MDM_1_MA_600_sinp_0.7071_tanb_1.0_DMsimpV2HDMa/metlog.pdf}
%    \caption{$E_{T}^{miss}$}
%  \end{subfigure}
%  \begin{subfigure}[b]{0.49\textwidth}
%    \includegraphics[width=\textwidth]{texinputs/04_grid/figures/DMHF/benchmarking/MDM_1_MA_600_sinp_0.7071_tanb_1.0_DMsimpV2HDMa/top1ptlog.pdf}
%    \caption{Leading top $p_{T}$}
%  \end{subfigure} \\
%  \begin{subfigure}[b]{0.49\textwidth}
%    \includegraphics[width=\textwidth]{texinputs/04_grid/figures/DMHF/benchmarking/MDM_1_MA_600_sinp_0.7071_tanb_1.0_DMsimpV2HDMa/top2ptlog.pdf}
%    \caption{Trailing top $p_{T}$}
%  \end{subfigure}
%  \caption{The $E_{T}^{miss}$, leading and trailing top $p_{T}$ distributions for inclusive $t\bar{t}+\chi\bar{\chi}$ production generated from the \texttt{DMsimp} (solid) and the 2HDM+a (dashed) models with various values of $\mathrm{M_a}$. The 2HDM+a models are generated with the following model parameters:$\mathrm{M_A}=600$ GeV, $\mathrm{M_H}=\mathrm{M_{H^{\pm}}}=600$ GeV, $\tan\beta=1$, and $\sin\theta=0.7071$.}
%\label{fig:DMSimpV2HDMa}
%\end{figure}
%%%%%-----------------------------------------------------------------------------------------
%\subsubsection{Recasting existing tt+\met and bb+\met signatures}
%\label{subsub:hfttrecast}
%These two signatures are dominantly produced in diagrams involving the invisible decays of the two CP-odd scalars. 
%Their relevance is therefore determined by the two pseudoscalar masses, $m(A)$ and $m(a)$ and it is a function of 
%$sin\theta$ and $tan\beta$. 
%For both $bb$ and $tt$ associated productions, we find that the
%highest sensitivity of this signatures is obtained for high values of $sin\theta$.
%
%The 2HDM+a model is equivalent to a single pseudoscalar simplified
%model (DMF) when $A$ is much heavier than $a$, and therefore the
%former does not contribute to the considered final state. However,
%when the two mediators are closer in mass, the $pp\rightarrow ttA$
%contribution becomes more relevant  as it is possible to observe in 
%Figure~\ref{fig:mdd}, where the two models are compared
%assuming $m(A) = 750$ GeV and two different values
%for $m(a)$. An excellent agreement was observed between
%$DMSIMP$ and $2HDMp$ on parton-level variables sensitive 
%to the helicity structure of the interaction between top and the
%mediator\cite{Haisch:2016gry}, if the invariant mass of the two DM
%particles in the 2HDM is required to be smalle than 200(300)~GeV for
%$m(a)=150(300)$~GeV respectively, giving confidence that,
%once the contribution from $A$ production is separated, it is possible
%to fully map the $2HDM+a$ kinematics into the DMF simplified model. 
%
%
%\begin{figure}[htb]
%\begin{center}
%\includegraphics[width=0.48\textwidth]{texinputs/04_grid/figures/DMHF/mdd150.pdf}
%\includegraphics[width=0.48\textwidth]{texinputs/04_grid/figures/DMHF/mdd300.pdf}
%\caption{Comparison of $m(\chi\chi)$, the invariant mass of 
%the two DM particles for the $DMSIMP$ (blue) and the $2HDMp$ model (magenta). The plot on the left (right) shows the comparison for $m(a)=150(300)$~GeV
%respectively.}
%\label{fig:mdd}
%\end{center}
%\end{figure}
%
%This remapping is achieved by
%taking for each set of the parameters the 
%average of the selection acceptances for $m(A)$ and $M(A)$ as 
%calculated with $DMSIMP$ weighted by the respective 
%cross-section for $A$ ($\sigma_A$) and $a$ 
%($\sigma_a$) production, in formulas
%\begin{equation}
%Acc_{2HDM}(m(A),M(a))=\frac{\sigma_a \times Acc_{DMSIMP}(m(a))+
%\sigma_A \times Acc_{DMSIMP}(m(A))}{\sigma_a+\sigma_A}
%\label{rew}
%\end{equation}
%The acceptance in this case is a parton level implementation of the
%two-lepton analysis described in [arXiv:1710.11412].
%The acceptance estimated in this way is shown as red triangles 
%in Figure~\ref{fig:tbfin}, and an excellent agreement 
%can be seen with the acceptances evaluated directly on the 2HDM 
%samples. 
%\begin{figure}[htb]
%\begin{center}
%\includegraphics[width=0.7\textwidth]{texinputs/04_grid/figures/DMHF/plotacc_tb.pdf}
%\caption{Acceptance of the two-lepton analysis as a function of $\tan\beta$ 
%for the $2HDMp$ model (round markers), for the $2HDMp$ model 
%considering only events with $m(\chi\chi)<200$~GeV (square markers),
%and for the $DMSIMP$ model (full line) for a mediator
%mass of 150~GeV. The two dashed lines indicate
%the statistical error of the $DMSIMP$. The value of $m(A)$ is fixed at 
%600~GeV, and $\sin\theta=0.35$. 
%The acceptance 
%calculated from the $DMSIMP$ acceptance rescaled following the 
%prescription \ref{rew} (red triangles) is also shown.}
%\label{fig:tbfin}
%\end{center}
%\end{figure}
%The acceptance estimated in this way is shown as red triangles 
%in Figure~\ref{fig:tbfin}, and an excellent agreement 
%can be seen with the acceptances evaluated directly on the 2HDM 
%samples. Further validation were performed also on the acceptances calculated for zero and one lepton final states [1710.11412,1711.11520], 
%both as a function of $sin\theta$ and $tan\beta$ and can be observed in Fig~\ref{DMHF:pof}.
%Finally, the formula was succesfully tested also the situation in
%which $|m(A)-m(a)| \sim 50$ GeV, 
%implying the possibility of a large interference
%between the production of the two bosons.
%
%\begin{figure}
%\includegraphics[width=.5\textwidth]{texinputs/04_grid/figures/DMHF/SRt2_600_150_sin}
%\includegraphics[width=.5\textwidth]{texinputs/04_grid/figures/DMHF/DM_high_600_150_tan}
%\caption{Validation of the re-scaling formula on zero and one lepton final states as a function of $tan\beta$ and $sin\theta$ parameters}
%\label{DMHF:pof}
%\end{figure}
%
%%%%-----------------------------------------------------------------------------------------
\subsubsection{Flavour scheme recommendations and studies}
\begin{figure} \centering
  \begin{subfigure}[b]{0.49\textwidth}           
    \includegraphics[width=\textwidth]{texinputs/04_grid/figures/DMHF/4v5flavour/MDM_1_Ma_100_MA_500_sinp_0.7071_tanb_1.0_4F_v_5F/metlog.pdf}
    \caption{$E_{T}^{miss}$}
  \end{subfigure}
  \begin{subfigure}[b]{0.49\textwidth}
    \includegraphics[width=\textwidth]{texinputs/04_grid/figures/DMHF/4v5flavour/MDM_1_Ma_100_MA_500_sinp_0.7071_tanb_1.0_4F_v_5F/topptlog.pdf}
    \caption{top $p_{T}$}
  \end{subfigure}
  \caption{$E_{T}^{miss}$ and top $p_{T}$ distributions for $M_{h_{4}}=100$ GeV, $M_{h_{3}}=500$ GeV, $M_{DM}=1$ GeV, $\mathrm{sin\theta}=0.7071$, and $\mathrm{tan\beta}=1$.}
  \label{fig:4v5_Ma100_MA500}
\end{figure}

\begin{figure} \centering
  \begin{subfigure}[b]{0.49\textwidth}           
    \includegraphics[width=\textwidth]{texinputs/04_grid/figures/DMHF/4v5flavour/MDM_1_Ma_200_MA_750_sinp_0.7071_tanb_1.0_4F_v_5F/metlog.pdf}
    \caption{$E_{T}^{miss}$}
  \end{subfigure}
  \begin{subfigure}[b]{0.49\textwidth}
    \includegraphics[width=\textwidth]{texinputs/04_grid/figures/DMHF/4v5flavour/MDM_1_Ma_200_MA_750_sinp_0.7071_tanb_1.0_4F_v_5F/topptlog.pdf}
    \caption{top $p_{T}$}
  \end{subfigure}
  \caption{$E_{T}^{miss}$ and top $p_{T}$ distributions for $M_{h_{4}}=200$ GeV, $M_{h_{3}}=750$ GeV, $M_{DM}=1$ GeV, $\mathrm{sin\theta}=0.7071$, and $\mathrm{tan\beta}=1$.}
  \label{fig:4v5_Ma200_MA750}
\end{figure}

\begin{figure} \centering
  \begin{subfigure}[b]{0.49\textwidth}           
    \includegraphics[width=\textwidth]{texinputs/04_grid/figures/DMHF/4v5flavour/MDM_1_Ma_200_MA_300_sinp_0.25_tanb_1.0_4F_v_5F/metlog.pdf}
    \caption{$E_{T}^{miss}$}
  \end{subfigure}
  \begin{subfigure}[b]{0.49\textwidth}
    \includegraphics[width=\textwidth]{texinputs/04_grid/figures/DMHF/4v5flavour/MDM_1_Ma_200_MA_300_sinp_0.25_tanb_1.0_4F_v_5F/topptlog.pdf}
    \caption{top $p_{T}$}
  \end{subfigure}
  \caption{$E_{T}^{miss}$ and top $p_{T}$ distributions for $M_{h_{4}}=200$ GeV, $M_{h_{3}}=300$ GeV, $M_{DM}=1$ GeV, $\mathrm{sin\theta}=0.25$, and $\mathrm{tan\beta}=1$.}
  \label{fig:4v5_Ma200_MA300}
\end{figure}

The relevant kinematic distributions for $t\bar{t}+\chi\bar{\chi}$ associated production in the context of this model are shown to be independent from the choice of PDF flavour scheme. In Figures~\ref{fig:4v5_Ma100_MA500}$-$\ref{fig:4v5_Ma200_MA300}, the $E_{T}^{miss}$, which is taken to be the $p_{T}$ of the $\chi\bar{\chi}$ system, and the $p_{T}$ distribution of the top quarks is presented using the 4 and 5-flavour scheme. The 4-flavour LHAPDF ID is 263400 and corresponds to \texttt{NNPDF30\_lo\_as\_0130\_nf\_4}, and the 5-flavour LHAPDF ID is 263000 and corresponds to \texttt{NNPDF30\_lo\_as\_0130}. As demonstrated for various configurations of the 2HDM+a model parameters, the kinematics are not affected by the flavour scheme choice of PDF. Furthermore, the difference in cross-section between the 4-flavour and 5-flavour generated LO $t\bar{t}+\chi\bar{\chi}$ process is at the $2-3$\% level, as noted in Tab.~\ref{tab:4v5_xsec}.

Despite the lack of kinematic dependence on flavour scheme, it is recommended to use the 5-flavour PDF.\textcolor{red}{Add support/discussion and references}

\begin{table}[htbp]
  \begin{tabular}{|c|c|c|c|c|c|c|}
    \hline
    $\mathrm{M_{h_{2}}},\mathrm{M_{h_{c}}}$ [GeV] & $\mathrm{M_{h_{3}}}$ [GeV] & $\mathrm{M_{h_{4}}}$ [GeV] & $\sin\theta$ & $\tan\beta$ & 4F $\sigma$ (pb) & 5F $\sigma$ (pb) \\
    \hline \hline
    750 & 500 & 100 & 0.7071 & 1 & 0.0988596 & 0.0964933 \\
    \hline
    750 & 750 & 200 & 0.7071 & 1 & 0.0445115 & 0.043149 \\
    \hline
    750 & 300 & 200 & 0.25 & 1 & 0.0310152 & 0.0300196 \\ 
    \hline
  \end{tabular}
  \caption{Configurations of the 2HDM+a model used to generate the $t\bar{t}+\chi\bar{\chi}$ process at LO and the corresponding cross-sections from the 4-flavour (4F) and 5-flavour (5F) PDF.}
  \label{tab:4v5_xsec}
\end{table}
%%%%-----------------------------------------------------------------------------------------
\paragraph{Motivations for an high $tan\beta$ scan for bb+\met}

The projection of sensitivity in $tan\beta$ for benchmark \#2,  based on the CMS results for bb+MET [arXiv:1706.02581]
are shown in Figure~\ref{DMHF:bbscan}. The reach for an upper bound on tan(beta) with bb+MET shows 
good potential, for $tan\beta$ values above 10. 

\textbf{Say something about high width for H?}

\begin{figure}
\includegraphics[width=.6\textwidth]{texinputs/04_grid/figures/DMHF//MAvsTB.pdf}
\caption{Sensitivity projection for benchmark \#2 based on the CMS results for bb+MET [arXiv:1706.02581].}
\label{DMHF:bbscan}
\end{figure}



%%%%-----------------------------------------------------------------------------------------
%%%%-----------------------------------------------------------------------------------------
%%%%-----------------------------------------------------------------------------------------
\clearpage
\subsubsection{Motivation for a dedicated tW+\met search}

The sensitivity of the LHC experiments to the associated 
production of dark matter with a single top has been recently studied \cite{Pani:2017qyd} in the framework
of an extension of the standard model featuring two Higgs doublets and
an additional pseudoscalar mediator. 
This study extends the work of previous literature \cite{Pinna:2017tay}, which demostrated using
a simplified model that the consideration of final states involving a single top quark and DM (DM$t$)
increases the coverage of existing analyses targeting the DM$t\bar t$ process.  

\begin{figure}
\begin{center}
\begin{subfigure}{.23\textwidth}\centering
\includegraphics[width=\textwidth]{texinputs/04_grid/figures/DMHF/Pfeyn_tw2}
\caption{}
\end{subfigure}
\begin{subfigure}{.23\textwidth}\centering
\includegraphics[width=\textwidth]{texinputs/04_grid/figures/DMHF/Pfeyn_tw1}
\caption{}
\end{subfigure}
\begin{subfigure}{.23\textwidth}\centering
\includegraphics[width=\textwidth]{texinputs/04_grid/figures/DMHF/Pfeyn_tchan_1}
\caption{}
\end{subfigure}
\begin{subfigure}{.23\textwidth}\centering
\includegraphics[width=\textwidth]{texinputs/04_grid/figures/DMHF/Pfeyn_tchan_2}
\caption{}
\end{subfigure}
\caption{Representative diagrams for $tW$ and $t$-channel production of DM in association with a single top quark.}
%($pp \rightarrow tj\chi\chi$)}
\label{fig:feyn1}
\end{center}
\end{figure}


Like single top production within the SM, the DM$t$ signature in the model
 receives  three different types of contributions at leading order (LO) in QCD. These are $t$-channel production, $s$-channel production and
associated production together with a $W$ boson ($tW$) (Fig.~\ref{fig:feyn1}).
When the decay $H^{\pm}\rightarrow W^{\pm} a$ is possible, the $H^{\pm}$ is produced on-shell, 
and the cross-section of $pp \rightarrow tW\chi\chi$, 
assuming  $H^{\pm}$ masses of a few hundred \GeV, is around one order of magnitude larger 
than the one for the same process in the simplified model. Moreover the production 
and cascade decay of a resonance yields kinematic signatures
which can be exploited to separate the signal from the SM background. 


Dedicated selections considering one and two lepton final states are
developed to assess the coverage in parameter space for this signature 
at a centre-of-mass energy of  $14$ TeV assuming an integrated
luminosity of 300~fb$^{-1}$ in Ref.~\cite{Pani:2017qyd}. 
Background and signal Monte Carlo simulated
samples are employed for the estimate of the results. The effect of the detector on the kinematic quantities
utilised in the analysis is simulated by applying a Gaussian smearing to the momenta
of  the different reconstructed objects and reconstruction and tagging efficiency factors.
Figure~\ref{DMHF:monotopres} shows the sensitivy reach for two of the
parameter scans proposed in this whitepaper. 
On the top panel the exclusion reach for the $m(a),tan\beta$ plane is
presented, assuming $sin\theta = 0.35$ and $m(A) = m(H^\pm) = m(H) =
500$ GeV. 
%The results are derived from the simulated samples using a
%re-scaling  procedure described in Ref.[IN PREPARATION]
It is possible to observe that for this scenario the sensitivity reach
is comparable to the one from the mono-h signature as presented in
Ref.~\cite{Bauer:2017ota}. On the bottom panel of
Figure~\ref{DMHF:monotopres} the signature's sensitivity to benchmark \#4
 is evaluated for the first time. 

\begin{figure}
\centering
\includegraphics[width=.48\textwidth]{texinputs/04_grid/figures/DMHF/SRrec2l_2DSCAN_a}
\includegraphics[width=.47\textwidth]{texinputs/04_grid/figures/DMHF/SR2la_ULscan4}
\caption{Exclusion reach for  benchmark \#2 (top) and benchmark \#4 (bottom), 
assuming $sin\theta = 0.35$ and $m(A) = m(H^\pm) = m(H) = 500$ GeV.}
\label{DMHF:monotopres}
\end{figure}


%%%%-----------------------------------------------------------------------------------------
%%%%-----------------------------------------------------------------------------------------
%%%%-----------------------------------------------------------------------------------------
\subsubsection{Uncovered signatures with $tt h+\met$}

As discussed in Section~\ref{sub:hfttrecast}, the production of the
heavy mediator $A$ gives a sizeable contribution to the $tt+\met$  
production cross section in the $2HDM+a$ model. This is also true for
the heavy $H$. When the decay of these mediators into the lightest
pseudoscalar $a$ is allowed, this decay process dominates over the
direct decay into $\chi\chi$. In symmetry with what happens for the
mono-h signature discussed in \cite{Bauer:2017ota}, for certain region
of parameter space the signatures $pp \rightarrow t\bar t A
\rightarrow t \bar t a h$ and $pp \rightarrow t\bar t H
\rightarrow t \bar t a Z$ become sizeable. For the former case, it can
be estimated from Fig. 12(b) of Ref.~\cite{Bauer:2017ota} that for
relatively small $m(A)$ the $pp \rightarrow t\bar t ah$ cross section
can be up to 30\% that of the $pp\rightarrow t \bar t \chi\chi$
process. The interplay between the parameters of the model, and
especially between the heavy higgs masses for these types of final
state render the phenomenology interesting and variegated, as can be
seen for example in the branching ratio study of Fig.~\ref{fig:brAHah}, although
further studies are needed to fully understand the interplay and the
complementarity between these $tth+\met$ channels and the traditional
heavy flavour dark matter searches. 

\begin{figure}
\centering
\includegraphics[width=.48\textwidth]{texinputs/04_grid/figures/DMHF//brA}
\includegraphics[width=.48\textwidth]{texinputs/04_grid/figures/DMHF//brH}
\caption{Example of the dependence of the $A$ and $H$ branching ratio into $ah$ as a function of some parameters of the 2HDM model.}
\label{fig:brAHah}
\end{figure}
%%%%-----------------------------------------------------------------------------------------

%%%%-----------------------------------------------------------------------------------------

%%%%-----------------------------------------------------------------------------------------
\FloatBarrier
\subsubsection{Top pair resonant searches}
Heavy (pseudo)scalar bosons with $M_{A/H}\ge2\mt$ and $\tanb \sim \mathcal{O}(1)$ decaying dominantly into top-quark pairs can be
searched for by studying the resulting \ttbar\ invariant mass
spectra. However, interference effects between the signal processes
and the SM \ttbar\ production distort the signal shape from a single
peak to a peak-dip structure \cite{Carena:2016npr}. The first search in this challenging decay channel was conducted recently, probing scalar and pseudoscalar masses between 500 and 650\GeV\ in a minimal 2HDM \cite{Aaboud:2017hnm}. A similar kinematic range could be probed if the result were re-interpreted in the context of the \hdma. Interference between
a loop-induced and a tree-level process cannot currently be simulated in \mg. To amend this problem, the same "Higgs\_Effective\_Couplings\_FormFactor"
approach \cite{ttinterfHFF} as adopted in \cite{Aaboud:2017hnm} is implemented in the UFO, replacing the loop production by an 
effective vertex. The predictions of the modified UFO for the case, in which the pseudoscalar mediator does not mix with the heavy pseudoscalar $A$ ($\sinp=0$), i.e. effectively decouples from the 2HDM Higgs sector, are compared to those for the minimal 2HDM. Excellent agreement is found in the invariant mass distributions of $A/H$ decaying into a top pair are shown in Fig.~\ref{fig:ttres_2HDMvs2HDMa}. As examples of how the sensitivity changes as a function of the parameters of the \hdma, the $M_{\ttbar}$ distributions of pseudoscalars decaying into \ttbar\ are presented in Fig~\ref{fig:ttres_2HDM_A}. Larger values of \tanb\ or \sinp\ are expected to yield lower sensitivities to $A\rightarrow\ttbar$ significantly while \ma\ almost only affects the contribution from $a\rightarrow\ttbar$, which becomes sizeable if \ma is close to $2\mt$.
\begin{figure}
\centering
\includegraphics[width=.48\textwidth]{texinputs/04_grid/figures/ttres/ttres_2HDMvs2HDMa_A.pdf}
\includegraphics[width=.48\textwidth]{texinputs/04_grid/figures/ttres/ttres_2HDMvs2HDMa_H.pdf}
\caption{$M_{\ttbar}$ distribution of the heavy (pseudo)scalar boson decaying into \ttbar\ with $\mA=\mH=600\GeV, \tanb=0.4$, $\sinp=1/\sqrt{2}$ and $M_a=100\GeV$ in comparison with the one from the generic 2HDM.}
\label{fig:ttres_2HDMvs2HDMa}
\end{figure}

\begin{figure}
\centering
\begin{subfigure}[b]{0.49\textwidth}
\includegraphics[width=\textwidth]{texinputs/04_grid/figures/ttres/ttres_2HDMa_A_tanb.pdf}
\caption{\tanb dependency with fixed $\sinp=1/\sqrt{2}$ and $\ma=100\GeV$}
\end{subfigure}
\begin{subfigure}[b]{0.49\textwidth}
\includegraphics[width=\textwidth]{texinputs/04_grid/figures/ttres/ttres_2HDMa_A_sinp.pdf}
\caption{\sinp dependency with fixed $\tanb=0.4$ and $\ma=100\GeV$}
\end{subfigure}
\begin{subfigure}[b]{0.49\textwidth}
\includegraphics[width=\textwidth]{texinputs/04_grid/figures/ttres/ttres_2HDMa_A_ma.pdf}
\caption{\ma dependency with fixed $\tanb=0.4$ and $\sinp=1/\sqrt{2}$}
\end{subfigure}
\caption{parameter dependency of signal $M_{\ttbar}$ distribution mediated by pseudosalars. The value of \mA is fixed at 600\GeV.}
\label{fig:ttres_2HDM_A}
\end{figure}
\FloatBarrier
%%%%-----------------------------------------------------------------------------------------

%%%%-----------------------------------------------------------------------------------------
%%%%-----------------------------------------------------------------------------------------
\subsubsection{Four tops final states}

The topology involving four top-quarks in the final state is a rare,
yet increasingly important signature, which will gain sensitivity and
attention with the enlargement of the dataset delivered by the LHC.  
In the attempt to perform a first characterisation of this topology,
we have studied the predicted cross-section for the four top final
state of this model for two sets of parameter choices. 

In Figure~\ref{DMHF-4top-scan1} we present the four top cross section
for the parameter choices of benchmark \#2, for an intermediate choice
of mass of the light pseudoscalar ($m(a) = 400$ GeV), as a function of
$tan\beta$. The total four-top production cross section, which
accounts for both SM and new physics (NP) contributions and is indicated
as $|SM+NP|^2$ in the legend, is compared with the production cross
section contributions separately due to SM and NP terms. 
This is achieved technically by setting a requirement on the number of
QCD and QED vertices in madgraph, as indicated in Table~\ref{tab-dmhf-4tops}.
Furthermore, the different contributions from on-shell production of
each CP-odd and CP-even mediators associated with a top pair and
decaying into a top pair is indicated. The dominant contribution is
driven by the on-shell production of $A$ and $H$ for all choices of
$tan\beta$ in this benchmark. 
In the lower panel of Figure~\ref{DMHF-4top-scan1}, the effect of the
interference term between the 2HDM+a and the SM is assessed, and is
found to have an impact almost always smaller than 5\% on the
inclusive cross-section. \textbf{Checking whether true for some
fiducial cuts, would be important to add statement or clarify that is
is not fully conclusive as it is only inclusive.}

In Figure~\ref{DMHF-4top-scan2} we present instead the cross-section
study for the parameter choices of benchmark \#3b, for $sin\theta
= \frac{1}{\sqrt{2}}$ and as a function of the light pseudoscalar mass. 
Very interestingly, for this parameter choices the cross-section is
quite independent of $m(a)$. As it can
be observed from the on-shell contribution breakdown, at the
low-end of the mass spectrum the $\ttbar+a$ production dominates, with a
peak at $400$ GeV due to the competition between
$a\rightarrow \chi\chi$ and $a\rightarrow \ttbar$ and the natural
decreasing of the cross section with the increase of $m(a)$.  The contribution of
$\ttbar+H$ and $\ttbar+A$ processes compensates the latter effect in
the higher end of the mass-spectrum, with the turn on starting around
$800$ GeV due to the competition between $A/H\rightarrow\ttbar$ and
cascade decays of the heavy higgses into the light pseudoscalar
mediator ($A\rightarrow ah/H\rightarrow aZ$). 
The little bump at 1 TeV is due to interference effects between the
three higgs mediators, which are all set to the
same mass for this parameter choice.  
The inclusive production cross-section of the 2HDM+a
model is also compared with the one obtained by the DMSimp pseudoscalar
implementation. Furthremore, as for the previous benchmark,  the impact of the SM interference
term on the inclusive cross-section is found to be very small
($<2\%$), except for $m(a)$ values close to the top theshold. 


\begin{table}
\begin{tabular}{ccm{50mm}}
\toprule
{\sc Madgraph} rule & Legend symbol & Details \\\midrule
\verb| p p > t t~ t t~ / a z h1 QED<=2|& $|SM+NP|^2$ & Four-top
production including both SM and NP contributions and their
interference. \\\midrule
\verb| p p > t t~ t t~ / a z h1 QCD<=2|& $|NP|^2$ & Four-top
production from NP processes, including interference terms among
$A,H,a$. \\\midrule
\verb| p p > t t~ t t~ / a z h1 QED<=0|& $|SM|^2$ & Four-top SM
production.\\
\bottomrule
\end{tabular}
\caption{Description of the specific MADGRAPH settings used to derive
the different curves of Figs~\ref{DMHF-4top-scan1}~and~\ref{DMHF-4top-scan2}.}
\label{tab-dmhf-4tops}
\end{table}

\begin{figure}
\centering
\begin{subfigure}[b]{0.8\textwidth}
\includegraphics[width=\textwidth]{texinputs/04_grid/figures/DMHF/4tops/WHP_final_tbscan.pdf}
\caption{}
\label{DMHF-4top-scan1}
\end{subfigure}
\begin{subfigure}[b]{0.8\textwidth}
\includegraphics[width=\textwidth]{texinputs/04_grid/figures/DMHF/4tops/WHP_final_mascan.pdf}
\caption{}
\label{DMHF-4top-scan2}
\end{subfigure}
\caption{Four-top cross section study for a subset of the parameter
space of benchmark \#2 (top) and \#3 (bottom). The different Standard
Model (SM) and New Physics (NP) contributions with and without
interference and the breakdown in terms of on-shell mediator
production is presented, following the notation of Table~\ref{tab-dmhf-4tops}. }
\end{figure}


\subsubsection{2HDM+scalar sensitivity studies}

The sensitivity of the tt+\met all-hadronic signature~\ref{Aaboud:2018hf}
to the Type-II model with two Higgs doublets plus a 
scalar portal to dark matter $S_1$ described in~\ref{Bell:2016ekl} has been investigated.
The signature has been chosen as is the one providing the best sensitivity across the majority of the 
mass range for $S_1$ which is most difficult to address. 

The choice of the parameters for this model largely reflects that of the $2HDM+p$ model.
In particular, the masses of the heavier scalar $S_2$ and the masses of the charged Higgs and the CP-odd
scalar have been set to 600 GeV. The value of the mixing angle has been set to $\cos{\theta}=0.35$. 
The mass of the light scalar $S_1$ is scanned between $200$ and $340$ GeV, and $\tan\beta$ is scanned 
between 0.2 and 1. Across the whole $\tan\beta$ and mass range considered the widths of both the light and the heavy 
scalars do not exceed $15\%$ of the mass of the corresponding particle. 

The procedure to extract the results follows closely the procedure described in Sec.~\ref{subsub:hfttrecast}, 
but with the light and heavy pseudo-scalars a and A replaced by the light and heavy scalars
 $S_1$ and $S_2$, respectively.
The formula to rescale the results of the acceptance is therefore turned into:

\begin{equation}
Acc_{2HDM}(m(S_1),M(S_2))=\frac{\sigma_{S_1} \times Acc_{DMSIMP}(m(S_{1}))+
\sigma_{S_2} \times Acc_{DMSIMP}(m(S_{2}))}{\sigma_{S_1}+\sigma_{S_2}}
\label{rewS}
\end{equation}


The validity of this re-scaling has been validated by comparing the acceptance 
to the simplified models to the acceptance to the $2HDM+S_1$ models, before and 
after applying the rescaling of Eq.~\ref{rewS}. The results are shown if Fig.~\ref{fig:rescS1} across the whole
mass range for the light scalar $S_1$ considered, for the minimum and maximum values of 
$\tan\beta$ taken into account. 
It has to be noted that the acceptance to simplified models and $2HDM+S_1$ is very similar even 
before applying the rescaling of Eq.~\ref{rewS}.
This is due to the fact that the ratio between the cross-section of $S_2$ and the cross-section of $S_1$
becomes appreciable (i.e. above sub-percent level) only for models with $m(S_1)>300$ GeV
 and $\tan\beta > 0.4$.\\

\begin{figure}
  \centering
  \includegraphics[width=0.6\textwidth]{texinputs/04_grid/figures/DMHF/THDMs/rescalingS1tgb02.pdf}
   \includegraphics[width=0.6\textwidth]{texinputs/04_grid/figures/DMHF/THDMs/rescalingS1tgb10.pdf}
\caption{Validation of the re-scaling formula as a function of $m(S_1)$ for the minimum and maximum values of $\tan{\beta}$ considered.}
 \label{fig:rescS1}
\end{figure}

Fig.~\ref{fig:resultsS1} shows the results of the recasting of the simplified model results in the context of the 
$2HDM+S_1$ model. For the choice of parameters made, models with scalar mediator between 100 and 360 GeV 
can be excluded for $\tan\beta = 0.2$, while for $\tan\beta = 1$ masses between 100 and 120 GeV are excluded.

\begin{figure}
  \centering
  \includegraphics[width=0.6\textwidth]{texinputs/04_grid/figures/DMHF/THDMs/resultsS1.pdf}
\caption{Expected and observed exclusion limits at $95\%$ C.L. obtained as a result of the recasting of the all-hadronic channel
in the context of the $2HDM+S_1$ model.}
  \label{fig:resultsS1}
\end{figure}

%\printbibliography


%\end{document}





\subsection{Other signatures}
%Ahim� sull?internal note sono stati di poche parole sulla parte di mono-jet per 2HDM+a.
%Comunque gran parte dei risultati che ho prodotto sono sintetizzati in https://indico.cern.ch/event/660808/contributions/2697161/attachments/1510819/2355991/2017.08.21.2HDMaMonojet.pdf.

\subsubsection{Monojet}

The search for events with at least one jet and large missing transverse momentum in the final states can be also interpreted in the context of the 2HDM+a model. In this scenario the light pseudo-scalar mediator which decays in DM particles can be radiated from heavy quark loops providing such a signature. This channel is able to probe a phase space with low $\tan(\beta)$ and high $\sin\theta$ in which the cross-sections of this kind of processes are enhanced.

\subsubsection{Resonant Production at Collider (Comparison)}

\begin{figure}
\begin{center}
%\vspace*{1.5cm} 
\includegraphics[width=0.4\textwidth]{texinputs/04_grid/figures/MonoHiggsAa.pdf}
\includegraphics[width=0.4\textwidth]{texinputs/04_grid/figures/MonoZHAa.pdf} \vspace{2em}\\
\includegraphics[width=0.4\textwidth]{texinputs/04_grid/figures/MonoHiggsS2S1.pdf}
\includegraphics[width=0.4\textwidth]{texinputs/04_grid/figures/MonoZAS12.pdf}
\caption{Feynman diagrams for resonant production signals signatures leading to mono-Z or mono-h.} 
\label{fig:feynresprod}
\end{center}
\end{figure}

The cross section for a resonant production process, with final state X, where a spin-0 resonance $S$ is produced and then decays, can be written as

\begin{eqnarray}
\sigma(p p \rightarrow S \rightarrow X) &=& \frac{\Gamma(S\rightarrow X)}{M \Gamma s} \sum_{i} C_{i} \Gamma(S\rightarrow i) = \frac{1}{M  s} \sum_{i} C_{i} \Gamma(S\rightarrow i) BR(S\rightarrow X)
\end{eqnarray}

where $i$ are the possible initial states, $C_i$ are weight factors that account for the protons PDFs and colour factors, and $s$ is the center of mass energy squared $s=(13TeV)^2$.

The values of the $C_i$ are as follows

\begin{eqnarray}
C_{gg} &=& \frac{\pi^2}{8} \int_{M^2/s}^1 \frac{dx}{x} g(x)g\left(\frac{M^2}{sx}\right)\\
C_{q\bar{q}} &=& \frac{4\pi^2}{9} \int_{M^2/s}^1 \frac{dx}{x}\left(q(x)\bar{q}\left(\frac{M^2}{sx}\right) + q\left(\frac{M^2}{sx}\right)\bar{q}(x)\right)
\end{eqnarray}

Assuming gluon fusion production to be the dominant one, the ratio of the cross sections for the scalar and pseudoscalar model for mono-higgs and mono-Z will be

\begin{eqnarray}
\frac{\sigma_S(p p \rightarrow S_2 \rightarrow \bar{\chi}\chi h)}{\sigma_P(p p \rightarrow A \rightarrow \bar{\chi}\chi h)} &=& \frac{\Gamma(S_2\rightarrow gg)}{\Gamma(A\rightarrow gg)} \frac{BR(S_2\rightarrow S_1 h)}{BR(A\rightarrow a h)} \frac{BR(S_1\rightarrow \bar{\chi}\chi)}{BR(a\rightarrow \bar{\chi}\chi)}\\
\frac{\sigma_S(p p \rightarrow A \rightarrow \bar{\chi}\chi Z)}{\sigma_P(p p \rightarrow H \rightarrow \bar{\chi}\chi Z)} &=& \frac{\Gamma(A\rightarrow gg)}{\Gamma(H\rightarrow gg)} \frac{BR(A\rightarrow S_1 Z)}{BR(H\rightarrow a Z)} \frac{BR(S_1\rightarrow \bar{\chi}\chi)}{BR(a\rightarrow \bar{\chi}\chi)}
\end{eqnarray}

The ideal situation to detect a mono-higgs/Z signal is to have such BR close to 1: if this is the case, the ratio of the signals becomes just the ratio of the widths

\begin{eqnarray}
\frac{\sigma_S(p p \rightarrow S_2 \rightarrow \bar{\chi}\chi h)}{\sigma_P(p p \rightarrow A \rightarrow \bar{\chi}\chi h)} &\sim& \frac{\Gamma(S_2\rightarrow gg)}{\Gamma(A\rightarrow gg)} \label{eq:sigmamonohapprox}\\
\frac{\sigma_S(p p \rightarrow A \rightarrow \bar{\chi}\chi Z)}{\sigma_P(p p \rightarrow H \rightarrow \bar{\chi}\chi Z)} &\sim& \frac{\Gamma(A\rightarrow gg)}{\Gamma(H\rightarrow gg)} \label{eq:sigmamonozapprox}
\end{eqnarray}

\begin{figure}
\begin{center}
%\vspace*{1.5cm} 
\includegraphics[width=0.7\textwidth]{texinputs/04_grid/figures/ratio.pdf}
\caption{Ratio $F_S/F_P$ as a function of the mass $M=M_A=M_{S_2}$.} 
\label{fig:resonantratio}
\end{center}
\end{figure}

the width for a scalar or pseudoscalar particle to gluons are:

\begin{eqnarray}
\Gamma(S\rightarrow gg) &=& \frac{g_S^2 \alpha_s^2 M}{16\pi^3} F_S\left(\frac{4m_t^2}{M^2}\right)\\
\Gamma(P\rightarrow gg) &=& \frac{g_P^2 \alpha_s^2 M}{16\pi^3} F_P\left(\frac{4m_t^2}{M^2}\right)
\end{eqnarray}
where
\begin{eqnarray}
F_S(x) &=& x |1+(1-x)\arctan^2\frac{1}{\sqrt{x-1}} |^2\\
F_P(x) &=& x |\arctan^2\frac{1}{\sqrt{x-1}} |^2
\end{eqnarray}

and $g_S = -y_t \sin\theta \epsilon_u$ for $S=S_2$, $g_P = y_t \epsilon_u$ for $A$ in the scalar model, and $g_S= y_t \epsilon_u$ for $S=H$, $g_P = y_t \cos\theta \epsilon_u$ for $A$ in the PS model. Note that the definition of the mixing angle is reversed in the scalar model comparing to the PS. So assuming equivalent mixing angle configurations, Eq. \ref{eq:sigmamonohapprox} and \ref{eq:sigmamonozapprox} reduce to

\begin{eqnarray}
\frac{\sigma_S(p p \rightarrow S_2 \rightarrow \bar{\chi}\chi h)}{\sigma_P(p p \rightarrow A \rightarrow \bar{\chi}\chi h)} &\sim& \frac{F_S(\frac{4m_t^2}{M^2})}{F_P(\frac{4m_t^2}{M^2})} <1 \\
\frac{\sigma_S(p p \rightarrow A \rightarrow \bar{\chi}\chi Z)}{\sigma_P(p p \rightarrow H \rightarrow \bar{\chi}\chi Z)} &\sim& \frac{F_P(\frac{4m_t^2}{M^2})}{F_S(\frac{4m_t^2}{M^2})} >1
\end{eqnarray}

The ratio $F_S/F_P$ is shown in Fig. \ref{fig:resonantratio} as a function of the mass $M=M_A=M_{S_2}$.

