\subsection{Scalar Model Description}

We shall consider a Dirac DM candidate, $\chi$, and expand the scalar
sector of the SM to include two Higgs doublets, $\Phi_1$ and
$\Phi_2$, in addition to a real singlet scalar field, $S$.  
Following the discussion in \citep{Bell:2017rgi}, we outline the scalar potential which controls the mixing between the CP even scalars, and the Yukawa structure which dictates the coupling of those scalars to the DM and SM fermions. 

It is convenient to rotate $\{\Phi_1,\Phi_2\}$ to the Higgs basis $\{
\Phi_h,\Phi_H \}$, which is defined such that only one of the two
doublets obtains a vev. Taking $\langle \Phi_H \rangle = 0$ and 
$\langle \Phi_h \rangle = v \sim 246$ GeV, the two Higgs doublets 
are then defined by
\bea
\Phi_h &=& \cos\beta \Phi_1 + \sin\beta \Phi_2 = \left(
\begin{array}{cc}
 G^+ \\
\frac{v + h + i G^0}{\sqrt{2}} \\
\end{array}
\right),\label{eq:alignh}\\
\Phi_H &=& -\sin\beta \Phi_1 + \cos\beta \Phi_2 = \left(
\begin{array}{cc}
 H^+ \\
\frac{ H + i A}{\sqrt{2}} \\
\end{array}
\right).
\eea


The scalar potential consists of the usual 2HDM potential, augmented by terms involving the new singlet scalar $S$.
We will assume that the scalar potential has a spontaneously broken $\mathbb{Z}_2$ symmetry for the particle $S$. This may arise naturally, for example, in the case where $S$ is part of a complex scalar charged under a dark $U(1)$ gauge group.  The scalar potential is thus given by
\begin{equation}
\hat{V}(\Phi_h,\Phi_H,S) = \hat{V}_{\mathsc{2hdm}}(\Phi_h,\Phi_H) + \hat{V}_S(S) + \hat{V}_{S\mathsc{2hdm}}(\Phi_h,\Phi_H,S), \label{eq:potential}
\end{equation}
where\nobreak
\bea
\hat{V}_{\mathsc{2hdm}}(\Phi_h,\Phi_H) &=& \hat{M}_{hh}^2 \Phi_h^\dagger \Phi_h + \hat{M}_{HH}^2 \Phi_H^\dagger \Phi_H +  (\hat{M}_{hH}^2 \Phi_H^\dagger \Phi_h + h.c.) + \frac{\hat{\lambda}_h}{2} (\Phi_h^\dagger \Phi_h)^2 + \frac{\hat{\lambda}_H}{2} (\Phi_H^\dagger \Phi_H)^2 \nonumber \\
&+&\hat{\lambda}_3 (\Phi_h^\dagger \Phi_h)(\Phi_H^\dagger \Phi_H) + \hat{\lambda}_4 (\Phi_H^\dagger \Phi_h)(\Phi_h^\dagger \Phi_H)  
+ \frac{\hat{\lambda}_5}{2} \left( (\Phi_H^\dagger \Phi_h)^2 + h.c.\right),\\
\hat{V}_S(S) &=& \frac{1}{2} \hat{M}_{SS}^2 S^2 + \frac{1}{4} \hat{\lambda}_S S^4,
\\
\hat{V}_{S\mathsc{2hdm}}(\Phi_h,\Phi_H,S) &=& 
 \frac{\hat{\lambda}_{HHS}}{2}(\Phi_H^\dagger \Phi_H)S^2 +  \frac{\hat{\lambda}_{hhs}}{2} \Phi_h^\dagger \Phi_h S^2 + \frac{1}{2}(\hat{\lambda}_{hHS} \Phi_H^\dagger \Phi_h S^2 + h.c.).
\eea

In general, there would be mass mixing between all three neutral CP even scalars of the model, $h$, $H$, and $S$. We shall, however, impose a generalized Higgs ``alignment limit" which decouples the SM Higgs, $h$, from the other two states. This is desirable because it reduces the scalar mixing to a 2-state problem and guarantees that $h$ couples like the SM Higgs. We thus set
\begin{align}
  \label{eq:align}  \hat{\lambda}_h &= \hat{\lambda}_H = \hat{\lambda}_3 + \hat{\lambda}_4 + \hat{\lambda}_5, \\
    \hat{\lambda}_{hhs} &= 0,
\end{align}
where the first of these conditions is sufficient to impose alignment in a standard 2HDM, and the second clearly prevents $h$-$S$ mixing (refer to \citep{Bell:2017rgi} for further discussion of the alignment limit).
%
The remaining $H$-$S$ mass matrix is then diagonalized to obtain two mass eigenstate scalars, $S_1$ and $S_2$, such that
\begin{align}
H &= \cos\theta S_1 - \sin\theta S_2,\\
S &= v_S + \sin\theta S_1 + \cos\theta S_2,
\end{align}
where 
\begin{equation}
\sin2\theta = \frac{2 \hat{\lambda}_{hHs} v v_S }{M_{S_1}^2-M_{S_2}^2}.
\end{equation}
The scalar mass spectrum then simplifies to
\begin{eqnarray}
%\label{eq:mh} M_{h}^2 &=& \hat{\lambda}_h v^2, \\
%M_{H^+}^2 &=&  M_{HH}^2+\frac{1}{2} \left(\hat{\lambda}_{HHs}+2\tan\beta \hat{\lambda}_{hHs}\right) v_S^2+\hat{\lambda}_{3}\frac{v^2}{2},\\
M_A^2 &=& M_{H^+}^2 + \left(
  \hat{\lambda}_4-\hat{\lambda}_5 \right)\frac{v^2}{2} ,\\
\label{MS12} M_{S_{1,2}}^2 &=&
\frac{1}{2}\left(M_A^2+\hat{\lambda}_5 v^2\right)\left(1\pm \frac{1}{\cos
  2\theta}\right) + \hat{\lambda}_S v_S^2\left(1\mp \frac{1}{\cos
  2\theta}\right). 
  %\\
%\tan2\theta &=& \frac{2 \hat{\lambda}_{hHs} v v_S }{M_A^2-2 \hat{\lambda}_S v_S^2+\hat{\lambda}_5 v^2} 
% \rightarrow \sin2\theta = \frac{2 \hat{\lambda}_{hHs} v v_S }{M_{S_1}^2-M_{S_2}^2}.
\end{eqnarray}
Taking $h$ to be the observed SM Higgs boson then fixes $\hat{\lambda}_h$ and hence also fixes $\hat{\lambda}_H$ via the alignment condition \eqref{eq:align}. Using the previous equations, we can rewrite $M_A$ and $M_{H^+}$ as a function of $M_{S_{1,2}}$, $\theta$, and $\hat{\lambda}_{4,5}$:
\begin{eqnarray}
M_{H^+}^2 &=&  M_{S_1}^2\cos^2\theta + M_{S_2}^2\sin^2\theta - \left(
  \hat{\lambda}_4+\hat{\lambda}_5 \right)\frac{v^2}{2},\\
M_A^2 &=& M_{S_1}^2\cos^2\theta + M_{S_2}^2\sin^2\theta -\hat{\lambda}_5 v^2.
\end{eqnarray}  

The only portal between the DM and other fields is via its Yukawa
coupling to the singlet scalar,
\begin{align}
\mathcal{L}_\mathsc{dm} &= -y_\chi S \bar{\chi} \chi.
\end{align}
We will assume that the DM particle has no bare mass term, and that its mass is instead generated by the vacuum expectation value of the singlet scalar, i.e. $m_\chi = y_\chi v_s$ with $\langle S \rangle = v_s$. Although this is not strictly necessary, such a scenario arises naturally if DM is a chiral fermion charged under some dark gauge group that is broken spontaneously by the vev of $S$. This assumption adds a constraint between the DM mass and the DM Yukawa coupling, removing the freedom to accommodate the relic density by varying these two parameters independently. 

\begin{table}[tb]\centering
\begin{tabular}{|c|c|c|c|}
\hline
Model & $\epsilon_d$ & $\epsilon_u$ & $\epsilon_l$
\\ \hline
Type I & $\cot\beta$ & $\cot\beta$ & $\cot\beta$
\\
Type II & $-\tan\beta$ & $\cot\beta$ & $-\tan\beta$
\\
Type X & $\cot\beta$ & $\cot\beta$ & $-\tan\beta$
\\
Type Y & $-\tan\beta$ & $\cot\beta$ & $\cot\beta$
\\
Inert & 0 & 0 & 0
\\ \hline
\end{tabular}
\caption{Values of the Yukawa scaling factors, $\epsilon_{u,d,l}$ which
  correspond to models with discrete ${\cal Z}_2$
  symmetries.}\label{tab:coeffs}
\end{table}

We will express the Yukawa interactions of the SM fermions with the Higgs doublets as
\begin{equation}
L_{\text{Yukawa}} = - \sum_{n=h,H} \left(Y_{n,ij}^U \bar{Q}_L^i u_R^j \widetilde{\Phi}_n
+ Y_{n,ij}^D \bar{Q}_L^i d_R^j \Phi_n
+ Y_{n,ij}^L \bar{L}_L^i l_R^j \Phi_n + h.c. \right),
\label{eq:yukawah1h2}
\end{equation}
and we will assume that the Yukawa matrices of the additional doublet are proportional to the SM ones: 
\begin{align}
Y_h^i & \equiv Y_{\mathsc{sm}}^i,\\
Y_H^i & = \epsilon_i Y_{\mathsc{sm}}^i,
\end{align}
where the $\epsilon_i$ are Yukawa scaling factors, with $i=u,d,l$. This Yukawa structure is the so-called Aligned Yukawa model \citep{Pich:2009sp,Tuzon:2010vt,Pich:2010ic,Penuelas:2017ikk,Gori:2017qwg}, which satisfies Natural Flavour Conservation. In special cases where the $\epsilon_i$ satisfy certain relationships, the Aligned Yukawa structure can correspond to one of the $Z_2$ symmetric Yukawa structures (Type I, II, X or Y), as shown in Table \ref{tab:coeffs}. While we will determine constraints for both Type I and Type II Yukawa structures, we will also include results for more general choices of the scaling factors that satisfy the Aligned Yukawa criteria. See \citep{Bell:2016ekl} for a more detailed discussion of the Yukawa structure in these models.