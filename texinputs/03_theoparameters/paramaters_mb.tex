\documentclass[12pt]{article}
\pdfoutput=1

\usepackage[a4paper,text={16.8cm,22.4cm}]{geometry}
\usepackage{amsmath,amsfonts,braket,slashed,amssymb,tikz,bm,psfrag,graphicx,color,dsfont,tikz}
\usepackage{multicol}
\usepackage{ctable}

\usepackage{cite}

\renewcommand{\Re}{\mathrm{Re}}

\renewcommand\({\left(}
\renewcommand\){\right)}
\renewcommand\[{\left[}
\renewcommand\]{\right]}




\begin{document}




\paragraph{Description of the model}
The 2HDM+a model is a two Higgs doublet model with two scalar doublets $H_1$ and $H_2$ and an additional pseudoscalar singlet $P$. It is the simplest renormalizable extension of the simplified pseudoscalar mediator model with SM singlet Dark Matter, which makes the gauge symmetry manifest by coupling to the DM (here a Dirac fermion $\chi$) with the singlet $P$
%
\begin{equation} \label{eq:Lx}
{\cal L}_\chi = - i \hspace{0.25mm} y_\chi P \hspace{0.25mm} \bar \chi \hspace{0.25mm} \gamma_5 \hspace{0.1mm} \chi \,,
\end{equation}
while the Higgs doublets couple to the SM fermions 
\begin{equation} \label{eq:LY}
{\cal L}_{Y} = - \sum_{i=1,2} \left ( \bar Q Y_u^i \tilde H_i u_R  + \bar Q Y_d^i H_i d_R   + \bar L Y_\ell^i H_i \ell_R  + {\rm h.c.}  \right ) \,.
\end{equation}
The dominant mediator of interactions between the dark sector and the SM fermions is a superposition of the CP-odd component of $H_1$, $H_2$ and $P$. 
We impose a $Z_2$ symmetry under which $H_1\to H_1$ and $H_2\to -H_2$, such that only one Higgs doublet appears in each operator in \eqref{eq:LY}. The different ways to construct these operators result in different Yukawa coupling structures and we will remain as general as possible regarding this choice. 
The $Z_2$ symmetry is the minimal condition necessary to guarantee the absence of flavour-changing neutral currents (FCNCs) at tree-level \cite{Glashow:1976nt,Paschos:1976ay} and such a symmetry is realized in many well-motivated complete ultraviolet theories in the form of supersymmetry, a $U(1)$ symmetry or  a discrete symmetry acting on the Higgs doublets. We further choose all parameters in the scalar potential real, such that CP eigenstates are identified with the mass eigenstates, two scalars $h$ and $H$, two pseudoscalars $a$ and $A$, and a charged scalar $H^\pm$. Under these conditions, the most general scalar potential can be written as 
\begin{align}
V(P,H)=V_H+V_{PH}+V_P\,,
\end{align}
with the potential for the two Higgs doublets
\begin{align}\label{eq:VH}
V_{H} & = \mu_1 H_1^\dagger H_1 + \mu_2 H_2^\dagger H_2 + \left ( \mu_3  H_1^\dagger H_2 + {\rm h.c.} \right ) + \lambda_1  \hspace{0.25mm} \big ( H_1^\dagger H_1  \big )^2  + \lambda_2  \hspace{0.25mm} \big ( H_2^\dagger H_2 \big  )^2 \notag \\
& \phantom{xx} +  \lambda_3 \hspace{0.25mm} \big ( H_1^\dagger H_1  \big ) \big ( H_2^\dagger H_2  \big ) + \lambda_4  \hspace{0.25mm} \big ( H_1^\dagger H_2  \big ) \big ( H_2^\dagger H_1  \big ) + \left [ \lambda_5   \hspace{0.25mm} \big ( H_1^\dagger H_2 \big )^2 + {\rm h.c.} \right ]  \,,
\end{align}
potential terms which connect doublets and singlets 
\begin{equation} \label{eq:VHP}
\begin{split}
V_{HP}  = P \left ( i \hspace{0.1mm} b_P  \hspace{0.1mm}  H_1^\dagger H_2 + {\rm h.c.} \right ) + P^2 \left (  \lambda_{P1}  \hspace{0.1mm}  H_1^\dagger H_1 +   \lambda_{P2}  \hspace{0.1mm}  H_2^\dagger H_2 \right )  \,,
\end{split} 
\end{equation}
and the singlet potential
\begin{equation} \label{eq:VP}
V_{P}  =  \frac{1}{2} \hspace{0.5mm} m_P^2  P^2 +  \lambda_P\,P^4 \,.
\end{equation}
Upon rotation to the mass eigenbasis, we trade the five dimensionful and eight dimensionless parameters in the potential  for physical masses and mixing angles and three quartic couplings
\begin{align}
\left\{ \,\,\begin{matrix}
\mu_1,\,\mu_2,\\[3pt]
\mu_3\,, m_P^2,\, b_P\\[3pt]
\lambda_1\,,\lambda_2\,,\lambda_3\,,\lambda_4\,,\lambda_5\\
\lambda_{P1}\,,\lambda_{P2} \,, \lambda_P
\end{matrix}\,\,\right\}\qquad \quad \longleftrightarrow \quad \qquad \left\{ \,\,\begin{matrix}
v,\, M_h,\,\cos(\beta-\alpha)\\[3pt]
M_a\,, M_A\,, M_H\,,M_{H^\pm}\\[3pt]
t_\beta\,, \cos(\theta)\,, \\[3pt]
\lambda_3\,,\lambda_{P1}\,,\lambda_{P_2}\,,\lambda_P
\end{matrix}\,\,\right\}\,.
\end{align}
Out of these parameters, the electroweak scale $v=246$ GeV and the mass of the SM-like CP-even mass eigenstate $M_h=125$ GeV are fixed. The mixing angle $\alpha$ between the CP-even scalars $h$ and $H$ is constrained by Higgs coupling strength measurements \cite{} and we show the allowed parameter space in the $\cos(\beta-\alpha)$ plane in  Fig.~\ref{fig:higgsfit} for the Yukawa sector of a 2HDM of type II.  For arbitrary values of $t_\beta=v_2/v_1$ only the limit $\cos(\beta-\alpha)\approx 0$ is allowed, for which the couplings of the CP-even state $h$ align with the couplings of the SM Higgs boson. For the analyses discussed in the remainder of this paper, we choose this so-called alignment limit and treat $t_\beta$ as a free parameter.
%%
\begin{figure}[t]
\includegraphics[width=\textwidth]{Figs/Higgsfit}
\caption{\label{fig:higgsfit} Parameter space allowed by a global fit to Higgs coupling strength measurements for (from left to right) a Yukawa sector of type I ($Y_u^1  = Y_d^1 = Y_\ell^1 =0$), type II ($Y_u^1 = Y_d^2 = Y_\ell^2 =0$),  type III ($Y_u^1 = Y_d^1 = Y_\ell^2 =0$), and type IV ($Y_u^1  = Y_d^2 = Y_\ell^1 =0$). }
\end{figure}
%%
Electroweak precision measurements constrain the splitting between the masses $M_H, M_A, M_a$ and $M_{H^\pm}$, since loops of spin-0 states modify the propagators of the electroweak gauge bosons at one-loop. For $M_H=M_{H^\pm}$ and $\cos(\beta-\alpha)=0$, these corrections vanish due to a custodial symmetry in the tree-level potential $V_H$ \cite{} and the masses of the CP-odd mass eigenstates can be treated as free parameters. This custodial symmetry is also present in $V_H$ if $M_A=M_{H^\pm}$ and $\cos(\beta-\alpha)=0$, but the presence of the pseudoscalar mixing term in $V_P$ softly breaks this symmetry. As a consequence, the pseudoscalar mixing angle $\theta$ and the mass splitting between $M_H$, $M_A$ and $M_a$ are constrained in this situation. In Fig.~\ref{fig: EWPTs}...
Flavour observables are mostly sensitive to corrections from one-loop exchanges of the charged scalar ${H^\pm}$, whose contributions to $b \to X_s \gamma$ \cite{Hermann:2012fc,Misiak:2015xwa,Czakon:2015exa} and $B_s-\bar B_s$ mixing \cite{Abbott:1979dt,Geng:1988bq,Buras:1989ui,Eberhardt:2013uba} lead to the strongest indirect constraints on $M_{H^\pm}$. Since the couplings of the charged scalar only depend on $t_\beta$, these constraints result in the bound $\tan \beta \gtrsim 0.8$ for $M_{H^\pm}=750$ GeV, independent of the choice of the Yukawa sector.\\
%%
\begin{figure}
\centering
\includegraphics[width=.5\textwidth]{Figs/EWPM}
\caption{\label{fig:EWPM}Values of $M_H$ and $M_a$ allowed by electroweak precision constraints for $\cos(\beta-\alpha)=0, M_{H^\pm}=M_A=750$ GeV, and varying values of the pseudoscalar mixing angle $\sin \theta =0.25, 0.3, 0.35, 0.4, 0.5$ , and maximal mixing angle $\sin\theta =1/\sqrt{2}\approx 0.71$. The parameter space below and to the left of the dashed contours is excluded. }
\end{figure}
%%
In addition to these constraints, the potential $V_H$ needs to give rise to a stable vacuum breaking the electroweak symmetry, whereas the parameters in $V_P$ need not introduce a vacuum expectation value for $P$, and scattering amplitudes should remain perturbative \cite{Gunion:2002zf,Barroso:2013awa} and unitary \cite{Kanemura:1993hm,Akeroyd:2000wc,Ginzburg:2005dt,Grinstein:2015rtl} up to the UV scale, where the 2HDM+a is UV completed by a more complete theory.  These conditions impose additional constraints on the quartic couplings in the potential, which are satisfied for $M_H, M_A, M_a \lesssim \mathcal{O}(1)$ TeV and $\lambda_3, \lambda_{P1}, \lambda_{P2}$ and $\lambda_P$ of $\mathcal{O}(1)$ as long as $t_\beta$ is not too much smaller than one. For the case of fixed\footnote{The singlet quartic $\lambda_P$ is entirely irrelevant for the phenomenology of the 2HDM+a.} $\lambda_3, \lambda_{P1}, \lambda_{P2}$, the stability condition therefore leads to additional constraints on the mixing angle $\theta$ and the masses. The parameter space of the 2HDM+a is strongly constrained, as summarized below
%
\newline
\begin{tabular}{cc c}
&&\\
$v, M_h, \cos(\beta-\alpha) $&$\longleftrightarrow$& fixed by Higgs measurements,\\[.3cm]
$M_{H^\pm}\,, $  &$\longleftrightarrow $& constrained by flavour observables,\\[.3cm]
$\sin(\theta)\,, M_H \,\,\text{or} \,\,M_A$  &$\longleftrightarrow $& constrained EWPM,\\[.3cm]
$\lambda_3, \lambda_{P1}, \lambda_{P2}\,,\lambda_P $ &$\longleftrightarrow $& constrained by stability, perturbativity and unitarity constraints\,.\\[.3cm]
&&
\end{tabular}
\newline
%
This leaves us with effectively three three free parameters from the potential, the Dark Matter mass and the coupling of the mediator to the DM candidate
\begin{align}
\big\{ m_\chi\,,\,\,M_a\,,\,\, t_\beta\,, \,\, M_H\,\,\text{or}\,\,M_A\,\,, \,\, y_\chi\,\big\}\,.
\end{align}
%In comparison,  the simplified model with a singlet mediator can be described with four parameters: the dark matter mass $m_{DM}$, the mass of the mediator $m_a$, the interaction strength between the mediator and the SM fermions $g_q$, and between the mediator and the DM candidate $g_\text{DM}$.

%\begin{thebibliography}{999}
%
%%\cite{Beltran:2010ww}
%\bibitem{Beltran:2010ww} 
%M.~Beltran, D.~Hooper, E.~W.~Kolb, Z.~A.~C.~Krusberg and T.~M.~P.~Tait,
%%``Maverick dark matter at colliders,''
%JHEP {\bf 1009}, 037 (2010)
%%  doi:10.1007/JHEP09(2010)037
%[arXiv:1002.4137 [hep-ph]].
%%%CITATION = doi:10.1007/JHEP09(2010)037;%%
%%286 citations counted in INSPIRE as of 25 Feb 2018
%
%%\cite{Fox:2011pm}
%\bibitem{Fox:2011pm} 
%P.~J.~Fox, R.~Harnik, J.~Kopp and Y.~Tsai,
%%``Missing Energy Signatures of Dark Matter at the LHC,''
%Phys.\ Rev.\ D {\bf 85}, 056011 (2012)
%doi:10.1103/PhysRevD.85.056011
%[arXiv:1109.4398 [hep-ph]].
%%%CITATION = doi:10.1103/PhysRevD.85.056011;%%
%%408 citations counted in INSPIRE as of 25 Feb 2018
%
%%\cite{Goodman:2010ku}
%\bibitem{Goodman:2010ku} 
%J.~Goodman, M.~Ibe, A.~Rajaraman, W.~Shepherd, T.~M.~P.~Tait and H.~B.~Yu,
%%``Constraints on Dark Matter from Colliders,''
%Phys.\ Rev.\ D {\bf 82}, 116010 (2010)
%%  doi:10.1103/PhysRevD.82.116010
%[arXiv:1008.1783 [hep-ph]].
%%%CITATION = doi:10.1103/PhysRevD.82.116010;%%
%%544 citations counted in INSPIRE as of 25 Feb 2018
%
%
%\end{thebibliography}

\end{document}
