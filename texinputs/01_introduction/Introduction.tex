%\paragraph{Reasoning behind this effort}

Dark matter (DM) is one of the main search targets for LHC~\cite{LHC2008} experiments (for a recent review, see e.g. Ref.~\cite{Kahlhoefer:2017dnp}). 

Based on the characteristics of DM as a weakly interacting massive particle (WIMP)~\cite{Bertone:2004pz}, the ATLAS~\cite{ATLAS2008} and CMS~\cite{CMS2008} experiments at the LHC have searched for DM particle candidates manifesting as particles that escape the detectors in the form of missing transverse momentum (\MET). 
The design of experimental searches for invisible particles can in principle be kept independent from specific theoretical models, reflecting the lack of hints on the exact particle nature of DM. 
However, theoretical benchmarks are necessary to sharpen the search sensitivity, to characterize a possible discovery and to define a theoretical framework for comparison with non-collider results. 

Supersymmetry has been the theoretical framework used as a benchmark for many particle dark matter searches at the LHC. 
Non-supersymmetric LHC search benchmarks evolved with time: at the start of data taking, Effective Field Theories (EFT) were used due to their relative model independence~\cite{Cao:2009uw,Beltran:2010ww,Goodman:2010yf,Bai:2010hh,Goodman:2010ku,Fox:2011pm}.  
Further developments towards simplified models, each representing simple, single processes encapsulating the phenomenology of LHC DM interactions using a small set of parameters, occurred towards the start of the second LHC run~\cite{Abercrombie:2015wmb}. 
The coherent adoption of these simplified models by the LHC collaborations focused the LHC DM experimental search program, especially in the presentation of its results and their comparison to direct detection (DD) and indirect detection (ID)~\cite{Boveia:2016mrp,Albert:2017onk,CMSSummary,ATLASSummary}. 
Throughout this time, the community has been aware of the shortcomings related to the simplicity of these benchmark models, in particular the lack of theoretical consistency at all scales for some of them~\cite{Bell:2015sza,Ko:2016zxg,Englert:2016joy,Kahlhoefer:2015bea} and their limited phenomenology leading to the relevance of only a small set of experimental signatures.  

With this whitepaper, we take a step beyond these simplified models by identifying an example benchmark model and its parameters to be tested by LHC searches, with the following characteristics: 

\begin{itemize}
\item the model should preferably be an extension of one of one of the simplified models already used by the experimental collaborations;
\item the model should still be generic enough to be used in the context of broader, more complete theoretical frameworks.  
\item the model should be gauge invariant;
\item the model should have a sufficiently varied phenomenology to encourage comparison of different experimental signatures and search for DM in new, unexplored signatures;
\item the model should be of interest beyond the DM community, so that it can link to existing efforts and that other direct and indirect constraints can be identified;
\end{itemize}

The model chosen in this whitepaper is a Two-Higgs-Doublet-Model (2HDM)~\cite{deFlorian:2016spz} [asked Higgs BSM people] with the addition of a pseudoscalar boson mediating the interaction between DM and SM, termed \hdma in the following. 
It is the simplest renormalisable extension of a 2HDM and includes a SM-singlet DM candidate~\cite{Ipek:2014gua,No:2015xqa,Goncalves:2016iyg,Bauer:2017ota,Tunney:2017yfp}. 
It adds an extended scalar sector to the pseudoscalar model in~\cite{Abercrombie:2015wmb,Buckley:2014fba}, and is gauge invariant. The first section of this whitepaper (\autoref{sec:evolution}) describes the evolution of theories for 
LHC DM searches and details the characteristics of this model. 
It also provides guidance on the choice of benchmark parameters to be used by LHC searches, from considerations of vacuum stability consideration. 

In~\autoref{sec:kinematics}, we investigate in detail the kinematic distributions of this model for three experimental signatures that are sensitive to this model: 
\begin{itemize}
\item the Higgs+\MET signature, where the \MET is produced by the decay of the pseudoscalar that couples to dark matter and the Higgs is produced either in association with this pseudoscalar or as a product of the decay of the second pseudoscalar; 
\item the Z+\MET signature, where the \MET is produced by the decay of the pseudoscalar that couples to dark matter and the Z boson is either produced in association with this pseudoscalar or radiated by the heavy Higgs boson; 
\item the $t\bar{t}$+\MET signature, where pseudoscalar that couples to dark matter is produced in association with a $t\bar{t}$ pair or radiated by one of the top quarks. Notably, the kinematic distributions of the simplified pseudoscalar model can be rescaled to correspond to the \hdma. 
\end{itemize}
The variation of the kinematic distributions with the parameters of the model is a useful indicator of the generality of the search strategy with respect to the specific choices of model parameters that will be adopted by the collaborations. 

The sensitivity of searches for this model in those signatures, determined at generator-level in~\autoref{sec:sensitivity}, surpasses that of the jet+\MET searches, providing a motivation to explore their parameter space beyond the $s-$channel simplified models in~\cite{Abercrombie:2015wmb}. Moreover, we identify a number of other existing and unexplored signatures that are sensitive to this model and can be studied using dedicated searches in the future.  

Finally, we investigate the connection of this model with the relic abundance of DM in the universe in~\autoref{sec:relic}, and sketch the path forward to determine the complementarity of collider searches with DD and ID searches in~\autoref{sec:DDandID}. 